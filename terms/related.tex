% !TeX spellcheck = en_GB
% !TeX encoding = UTF-8

\section{Related work}

\paragraph{Term rewriting}
The field of term rewriting is a rich area of research.
For the purpose of this thesis, I would like to restrict the analysis of related work to Isabelle formalizations.
Most notably in this space is the \isafor/\ceta\ project: ``An Isabelle/HOL Formalization of Rewriting for Certified Tool Assertions''.%
\footnote{\url{http://cl-informatik.uibk.ac.at/isafor/}}
The project consists of multiple modules, some of which are available in the Archive of Formal Proofs:

\begin{description}
  \item[Abstract Rewriting]
    Sternagel and Thiemann~\cite{sternagel2010rewriting} give abstract characterizations of term rewriting systems, in particular definitions of properties like completeness and normalization.
    There is no fixed type of terms that are prescribed for rewriting; all definitions operate on arbitrary relations.
  \item[First-Order Terms]
    The same authors~\cite{sternagel2018terms} provide a general definition of first-order terms, i.e., composed of variables and function applications.
    Based on this, they introduce matching and unification algorithms.
  \item[Z Property]
    Felgenhauer \etal~\cite{felgenhauer2016z} formalized the \emph{Z property} based on work by Dehornoy and van Oostrom~\cite{dehornoy2008z}.
    As an example application, they prove that lambda calculus has the Church--Rosser property~\cite{church1936conversion}.
    Unfortunately, this formalization of higher-order terms does not include the notion of \emph{constants}.
\end{description}

\paragraph{Term algebras}
Schmidt-Schauß and Siekmann~\cite{schmidt1988unification} discuss the concept of \emph{unification algebras}.
They generalize terms to \emph{objects} and substitutions to \emph{mappings}.
A unification problem can be rephrased to finding a mapping such that a set of objects are mapped to the same object.
The advantage of this generalization is that other -- superficially unrelated -- problems like solving algebraic equations or querying logic programs can be seen as unification problems.

In particular, the authors note that among the similarities of such problems are that ``objects [have] variables'' whose ``names do not matter'' and ``there exists an operation like substituting objects into variables''~\cite[§1]{schmidt1988unification}.
The major difference between my formalization in §\ref{sec:terms:algebra} and the work by Schmidt-Schauß and Siekmann is that I use concrete types for variables and mappings.
Otherwise, some similarities to here can be found.

Eder~\cite{eder1985properties} discusses properties of substitutions with a special focus on a partial ordering between substitutions.
However, Eder constructs and uses a concrete type of first-order terms, similarly to Sternagel and Thiemann~\cite{sternagel2018terms}.

Williams~\cite{williams1991instantiation} defines substitutions as elements in a monoid.
In this setting, instantiations can be represented as \emph{monoid actions}.
Williams then proceeds to define -- for arbitrary sets of terms and variables -- the notion of \emph{instantiation systems,} heavily drawing on notation from Schmidt-Schauß and Siekmann.
Some of the presented axioms~\cite[§2]{williams1991instantiation} are also present in my formalization.
Consequently, generic theorems can be derived from those axioms:
for example \cref{thm:terms:algebra:disjoint},\footnote{Corollary 3.8B: $t\sigma = t$, if $\mathrm{var}(t) \cap \mathrm{dom}(a) = \emptyset$}
\cref{thm:terms:algebra:subst-restrict},\footnote{Corollary 3.8A: $t\sigma = t(\sigma|\mathrm{var}(t))$} or
\cref{thm:terms:algebra:subst-twice}.\footnote{Corollary 3.8D: $\sigma + \tau = \sigma \tau$, if $\mathrm{cdm}(\sigma) \cap \mathrm{dom}(\tau) = \emptyset$}

\paragraph{Higher-order terms}
Blanchette \etal\ have formalized $\lambda$-free higher-order terms in Isabelle~\cite{becker2016lambda,blanchette2016lambda}.
Their main application are term orders that are used in automated theorem proving.
The formalization is relevant insofar as their term type can be made an instance of the generic \holtype{term} class.\thyrefafp{Higher_Order_Terms}{Lambda_Free_Compat}
