% !TeX spellcheck = en_GB
% !TeX encoding = UTF-8

\section{Term types}
\label{sec:terms:types}

\authorship{
  Portions of this section are based on the publications \citetitle{hupel2018compiler}, authored by \citeauthor{hupel2018compiler}~\cite{hupel2018compiler}, and the AFP entry \citetitle{hupel2019algebra}, authored by \citeauthor{hupel2019algebra}~\cite{hupel2019algebra}.
}

\begin{code}
  \begin{subfigure}[t]{.4\linewidth}
    \begin{lstlisting}[language=Isabelle]
datatype nterm =
  Nconst string |
  Nvar string |
  Nabs string nterm |
  Napp nterm nterm\end{lstlisting}\holtypeindex{nterm}
    \caption{Explicit bound variable names}
    \label{code:terms:intermediate:named}
  \end{subfigure}\hfill
  \begin{subfigure}[t]{.55\linewidth}
    \begin{lstlisting}[language=Isabelle]
datatype pterm =
  Pconst string |
  Pvar string |
  Pabs ((term $\times$ pterm) set) |
  Papp pterm pterm\end{lstlisting}\holtypeindex{pterm}
    \caption{Explicit pattern matching}
    \label{code:terms:intermediate:pabs}
  \end{subfigure}

  \begin{subfigure}[t]{.4\linewidth}
    \begin{lstlisting}[language=Isabelle]
datatype sterm =
  Sconst string |
  Svar string |
  Sabs ((term $\times$ sterm) list) |
  Sapp sterm sterm\end{lstlisting}\holtypeindex{sterm}
    \caption{Ordered clauses}
    \label{code:terms:intermediate:seq}
  \end{subfigure}\hfill
  \begin{subfigure}[t]{.55\linewidth}
    \begin{lstlisting}[language=Isabelle]
datatype value =
  Vconstr string (value list) |
  Vabs ((term $\times$ sterm) list)
       (string $\holmap$ value) |
  Vrecabs (string $\holmap$ ((term $\times$ sterm) list))
          string
          (string $\holmap$ value)\end{lstlisting}\holtypeindex{value}
    \caption{Values}
    \label{code:terms:intermediate:value}
  \end{subfigure}
  \caption{Intermediate term types}
  \label{code:terms:intermediate}
\end{code}

\noindent
The original \holtype{term} type was already given in \cref{code:background:rewriting:term}.
The intermediate types are depicted in \cref{code:terms:intermediate}.
In this section, I will briefly describe them and explain how they differ from each other.
The proofs of the term algebra laws (§\ref{sec:terms:algebra}) are mostly technical, so I omit them here.

\subsection{de~Bruijn terms (\holtype{term})}
\label{sec:terms:types:term}

\begin{code}
  \begin{lstlisting}[language=Isabelle]
definition shift_nat :: nat $\Rightarrow$ int $\Rightarrow$ nat where
shift_nat //n// //k// = ||if|| //k// $\geq$ $0$ ||then|| //n// + nat //k// ||else|| //n// - nat |//k//|

fun incr_bounds :: int $\Rightarrow$ nat $\Rightarrow$ term $\Rightarrow$ term where
incr_bounds //inc// //lev// (Bound //i//) =
  (if //i// $\geq$ //lev// then Bound (shift_nat //i// //inc//) else Bound //i//)
incr_bounds //inc// //lev// ($\Lambda$ //u//) = $\Lambda$ incr_bounds //inc// (//lev// + $1$) //u//
incr_bounds //inc// //lev// ($t_1$ $\$$ $t_2$) = incr_bounds //inc// //lev// $t_1$ $\$$ incr_bounds //inc// //lev// $t_2$
incr_bounds _ _ //t// = //t//

fun replace_bound :: nat $\Rightarrow$ term $\Rightarrow$ term $\Rightarrow$ term where
replace_bound //lev// (Bound //i//) //t// =
  ||if|| //i// < //lev// ||then|| Bound //i//
  ||else if|| //i// = //lev// ||then|| incr_bounds (int //lev//) $0$ //t//
  ||else|| Bound (//i// - $1$)
replace_bound //lev// ($t_1$ $\$$ $t_2$) //t// = replace_bound //lev// $t_1$ //t// $\$$ replace_bound //lev// $t_2$ //t//
replace_bound //lev// ($\Lambda$ //u//) //t// = $\Lambda$ replace_bound (//lev// + $1$) //u// //t//
replace_bound _ //t// _ = //t//

abbreviation $\beta$_reduce :: term $\Rightarrow$ term $\Rightarrow$ term (_ [_]) where
//t// [//u//] $\equiv$ replace_bound $0$ t u\end{lstlisting}\index{beta-reduce@{$\beta\_\holconst{reduce}$ (constant)}}\index{beta-reduction@$\beta$-reduction}
  \caption{$\beta$-reduction on \holtype{term}s using index shifting}
  \label{code:terms:types:beta}
\end{code}

\holtypeindex{term}
The definition of \holtype{term} is almost an exact copy of Isabelle's internal term type, with the notable omissions of type information (§\ref{sec:background:rewriting}).
The implementation of $\beta$-reduction is straightforward via index shifting of bound variables (\cref{code:terms:types:beta}).
Substitution of free variables does not require any additional assumptions, because capture is impossible.
The full instantiation of the \holtype{term} class has already been given in \cref{code:terms:algebra:inst}.

\subsection{Explicit names (\holtype{nterm})}

\holtypeindex{nterm}
The \holtype{nterm} type is similar to \holtype{term}, but removes the distinction between \emph{bound} and \emph{free} variables (\cref{code:terms:intermediate:named}).
Instead, there are only named variables.
To avoid capture during substitution, there is an additional constraint that only closed terms may be substituted into terms.
This is reflected in the precondition of \cref{thm:terms:algebra:subst-vars} of the \holclass{term} class.

Looking at the bigger picture, this is reasonable, because substitutions are only performed with closed terms.
However, for proofs, it might still be necessary, so appropriate preconditions need to control when substitution is safe.

The instantiation \holtypejudgement{\holtype{nterm}}{\holclass{term}} is given below:
%
\begin{align*}
  \holconst{const} &= \holconst{Nconst}  &  \holconst{consts}\;(\holconst{Nabs}\;x\;t) &= \holconst{consts}\;t \\
  \holconst{free} &= \holconst{Nvar} & \holconst{frees}\;(\holconst{Nabs}\;x\;t) &= \holconst{frees}\;t - \{x\} \\
  \holconst{app} &= \holconst{Napp} & \holconst{subst}\;\sigma\;(\holconst{Nabs}\;x\;t) &= \holconst{Nabs}\;(\holconst{subst}\;(\sigma - x)\;t) \\
  \omit\rlap{$\holconst{abs\_pred}\;P\;t = (\forall x\;t'.\;t = \holconst{Nabs}\;x\;t' \implies P\;t' \implies P\;t)$}
\end{align*}

\noindent
The main difference to the \holtype{term} type is that the bound variable name affects substitution, i.e., it needs to be dropped from the environment.

\subsection{Explicit pattern matching (\holtype{pterm})}
\label{sec:terms:types:pterm}
\holtypeindex{pterm}

Functions in Isabelle are usually defined using \emph{implicit} pattern matching, that is, the terms $p_i$ occurring on the left-hand side $f\;p_1\;\ldots\;p_n$ of an equation may be constructor patterns.
This is also common among functional programming languages like Haskell or OCaml.
CakeML only supports \emph{explicit} pattern matching using case expressions.
A function definition consisting of multiple defining equations must hence be translated to the form $f = \lambda x.\;\holkeyword{case}\;x\;\holkeyword{of}\;\ldots$.
The elimination proceeds by iteratively removing the last parameter in the block of equations until none are left (§\ref{sec:intermediate:elim}).

In my formalization, I have decided to conflate the notion of ``abstraction'' and ``case expression'', yielding \emph{case abstractions,} represented as the \holconst{Pabs} constructor in \cref{code:terms:intermediate:pabs}.
This is similar to the \texttt{fn} construct in Standard ML, which denotes an anonymous function that immediately matches on its argument~\cite{milner1997definition}.
The same construct also exists in Haskell with the \texttt{LambdaCase} language extension in Haskell.\footnote{\url{https://downloads.haskell.org/~ghc/7.8.4/docs/html/users_guide/syntax-extns.html}}
As in §\ref{sec:terms:algebra:thy:matching}, the pairs of patterns and terms are referred to as \emph{clauses}.

I chose this representation for these reasons:
\begin{enumerate}
  \item
    It allows for a simpler language grammar because there is only one (shared) constructor for abstraction and case expression.
  \item
    The elimination procedure outlined above does not need to introduce fresh names in the process.
    Later, when translating to CakeML syntax, fresh names are introduced and proved correct in a separate step (§\ref{sec:intermediate:cakeml}).
\end{enumerate}

\noindent
Furthermore, observe that the clauses are unordered and are hence a set.
This stems from the unordered nature of the term-rewriting semantics (\cref{code:background:rewriting:semantics}): if a rule matches, it can be applied, no matter what order the rules were defined in.

As a short-hand notation, $\Lambda\{ p_1 \Rightarrow t_1, p_2 \Rightarrow t_2, \ldots \}$ represents $\holconst{Pabs}\;\{ (p_1, t_1), (p_2, t_2), \ldots \}$.

The instantiation \holtypejudgement{\holtype{pterm}}{\holclass{term}} is a bit more complicated than for \holtype{nterm}, because it has to take multiple clauses into account:
\begin{align*}
  \holconst{consts}\;(\holconst{Pabs}\;\cs) &= \bigcup\limits_{(p, t) \in \cs} \holconst{consts}\;t \\
  \holconst{frees}\;(\holconst{Pabs}\;\cs) &= \bigcup\limits_{(p, t) \in \cs} \holconst{frees}\;t - \holconst{frees}\;p \\
  \holconst{subst}\;\sigma\;(\holconst{Pabs}\;\cs) &= \holconst{Pabs}\;\{ (p, \holconst{subst}\;(\sigma - \holconst{frees}\;p)\;t) \mid (p, t) \in \cs \} \\
  \holconst{abs\_pred}\;P\;t &= (\forall \cs.\;t = \holconst{Pabs}\;\cs \implies (\forall\;p\;t.\;(p, t) \in \cs \implies P\;t) \implies P\;t)
\end{align*}
%
As for \holtype{nterm}s, substitution is only supported for closed terms.

\subsection{Ordered clauses (\holtype{sterm})}
\label{sec:terms:types:sterm}
\holtypeindex{sterm}

For CakeML, the clauses need to be applied in a deterministic order, i.e., sequentially.
The \holtype{sterm} type only differs from \holtype{pterm} by using \holtype{list} instead of \holtype{set} (\cref{code:terms:intermediate:seq}).
Hence, case abstractions use list brackets: $\Lambda[p_1 \Rightarrow t_1, p_2 \Rightarrow t_2, \ldots]$.

The instantiation \holtypejudgement{\holtype{sterm}}{\holclass{term}} is largely similar to above.

The proofs of \cref{thm:terms:algebra:subst-vars,thm:terms:algebra:subst-consts} can be shared between \holtype{sterm} and \holtype{pterm} with a simple trick.
First, I define a function that converts from \holtype{sterm} to \holtype{pterm}; call it \holconst{sterm\_to\_pterm} (the implementation is given in \cref{code:intermediate:seq:convert}).
This is always possible, because one can always discard an ordering, i.e., turn a list into a set.
Next, I prove that this conversion function is total and that the free variables and constants are preserved by it.
Finally, the following lemma allows to transfer results from \holtype{sterm} to \holtype{pterm}:
%
\begin{lemma}
  \[ \holconst{subst}\;(\holconst{map}\;\holconst{sterm\_to\_pterm}\;\sigma)\;(\holconst{sterm\_to\_pterm}\;t) = \holconst{sterm\_to\_pterm}\;(\holconst{subst}\;\sigma\;t) \]
\end{lemma}

\noindent
Taken to its conclusion, this trick could eventually be used to define \holtype{pterm} as a quotient type of \holtype{sterm}~\cite{huffman2013lifting}.

\remark*{
  For all terms that do not contain abstractions, \holconst{sterm\_to\_pterm} coincides with \holconst{convert\_term}.
  This applies to all conversion functions between term types that are used in the compiler (§\ref{sec:compiler}).
}

\subsection{Irreducible terms (\holtype{value})}
\label{sec:terms:types:value}
\holtypeindex{value}

CakeML distinguishes between \emph{expressions} and \emph{values}.
Whereas expressions may contain free variables or $\beta$-redexes, values are closed and fully evaluated.
Both have a notion of abstraction, but values differ from expressions in that they contain an environment binding free variables.

Consider the expression $(\lambda x. \lambda y. x)\;(\lambda z. z)$, which can be rewritten to $\lambda y. \lambda z. z$ by $\beta$-reduction.
Note how the bound variable $x$ disappears, since it gets substituted.
This is contrary to how programming languages are usually implemented:
Evaluation does not happen by substituting the argument term $t$ for the bound variable $x$, but by recording the binding $x \mapsto t$ in an environment \cite{kumar2014cakeml}.
A pair of an abstraction and an environment is usually called a \keyword{closure}~\cite{landin1964mechanical,turner2013history}.
Similarly to \holtype{pterm} and \holtype{sterm}, pattern matching is immediate, which is why the body of a closure is a list of clauses.

In CakeML, this means that evaluation of the above expression results in the closure $(\lambda y. x, \left[\texttt{"x"} \mapsto (\lambda z. z, []) \right])$.
Note the nested structure of the closure, whose environment itself contains a closure.

To reflect this in the formalization, I introduce a \holtype{value} type (\cref{code:terms:intermediate:value}) which distinguishes between:
\begin{induction}
  \item[Vconstr] constructor values: a data constructor applied to multiple values
  \item[Vabs] closures: clauses combined with an environment mapping variables to values
  \item[Vrecabs] recursive closures: a group of mutually recursive function bodies with an environment mapping variables to values
\end{induction}

\noindent
The above example evaluates to the closure
%
\[ \holconst{Vabs}\;\big[\embed y \Rightarrow \embed x\big]\;\big[\texttt{"x"} \mapsto \holconst{Vabs}\;[\embed z \Rightarrow \embed z]\;[]\big] \]
%

\paragraph{Recursive closures}\index{closure, recursive}
The third case for recursive closures only becomes relevant late in the compiler (§\ref{sec:intermediate:ml}).
In the initial term-rewriting semantics, there is a clear distinction between constants and variables.
Constants and their definitions are recorded in the rule set \rs.
Consequently, recursive calls are straightforward:
The appropriate definition for the constant can be looked up there.
CakeML knows no such distinction between constants and variables, hence everything has to reside in a single environment $\sigma$.

This can be illustrated with the following example.
Consider the defining equations of \holconst{odd} and \holconst{even}:
\begin{align*}
  \holconst{odd}\;0 &= \holconst{False} \\
  \holconst{odd}\;(\holconst{Suc}\;n) &= \holconst{even}\;n \\
  \holconst{even}\;0 &= \holconst{True} \\
  \holconst{even}\;(\holconst{Suc}\;n) &= \holconst{odd}\;n
\end{align*}

\noindent
When evaluating the term $\holconst{odd}\;k$, the definitions of \holconst{even} and \holconst{odd} themselves must be available in the environment captured in the definition of \holconst{odd}.
However, there is no way in Isabelle to inductively construct such a \holconst{Vabs} that refers to itself, because that would be an infinite object.
Instead, the expressions used to define \holconst{odd} and \holconst{even} are captured in a recursive closure.
There might be an alternative design using coinduction, but I have chosen to follow the same path as CakeML, where it is modelled in a similar way as here.

For the above example, this would result in the following global environment:
%
\[
  [\texttt{"odd"} \mapsto \holconst{Vrecabs}\;\css\;\texttt{"odd"}\;[], \texttt{"even"} \mapsto \holconst{Vrecabs}\;\css\;\texttt{"even"}\;[]]
\]
\[
  \begin{aligned}
    \text{where } \css = [&\texttt{"odd"} \mapsto [ \embed0 \Rightarrow \embed{\holconst{False}}, \embed{\holconst{Suc}\;n} \Rightarrow \embed{\holconst{even}\;n}], \\
                          &\texttt{"even"} \mapsto [ \embed0 \Rightarrow \embed{\holconst{True}}, \embed{\holconst{Suc}\;n} \Rightarrow \embed{\holconst{odd}\;n}]]
  \end{aligned}
\]

\noindent
Note that in the first line, the right-hand sides are values, but in \css{}, they are expressions.
The additional string argument denotes the \emph{active} function\index{function, active}.
The semantics then takes care to add those to the environment as appropriate.

\subsubsection{Conversions}
\label{sec:terms:types:value:conv}

The lack of a suitable \holclass{term} instantiation notwithstanding, it is possible to establish a relationship with \holtype{sterm}s by defining conversion functions.
Similar to how \holconst{linear} characterizes \holtype{term}s that lie within the pattern fragment, \holconst{is\_value}\holconstindex{is\_value} characterizes \holtype{sterm}s that lie within the value fragment.
%
\[
  \inferrule*{
  }{\holconst{is\_value}\;(\Lambda\;\cs)} \quad
  \inferrule*{
    \mathit{name} \in \C \\
    \forall t \in \mathit{ts}.\; \holconst{is\_value}\;t
  }{\holconst{is\_value}\;(\listcomb{\mathit{name}}{\mathit{ts}})}
\]
%
\noindent
Informally, a term that satisfies that predicate is referred to as a \keyword{term-value}.
Observe that this definition is parametrized on \C, which is the set of all constructor names in any given theory.
This set is automatically generated during deep embedding (§\ref{sec:deep:rel}).

The actual conversion from values to terms can be described as executing the substitution from the captured environments:
%
\begin{lstlisting}
fun value_to_sterm :: value $\Rightarrow$ sterm where
value_to_sterm (Vconstr //name// //vs//) =
  list_comb (Sconst //name//) (map value_to_sterm //vs//)
value_to_sterm (Vabs //cs// $\sigma$) =
  $\Lambda$ [(//pat//, subst (map value_to_sterm $\sigma$ $-$ frees pat) //t//)) | (//pat//, //t//) $\leftarrow$ //cs//]
value_to_sterm (Vrecabs //css// //name// $\sigma$) =
  $\Lambda$ [(//pat//, subst (map value_to_sterm $\sigma$ $-$ frees pat) //t//)) | (//pat//, //t//) $\leftarrow$ //css// //name//]
\end{lstlisting}\holconstindex{value\_to\_sterm}

\noindent
The opposite direction does not need to perform any substitution, but it has to traverse a term according to its applicative structure (§\ref{sec:terms:algebra:thy:applicative}):
%
\begin{lstlisting}
fun sterm_to_value :: sterm $\Rightarrow$ value where
sterm_to_value //t// = ||case|| strip_comb //t// ||of||
  (Sconst //name//, //args//) $\Rightarrow$ Vconstr //name// (map sterm_to_value //args//)
  (Sabs //cs//, []) $\Rightarrow$ Vabs cs [])
\end{lstlisting}\holconstindex{sterm\_to\_value}
%
\noindent
Note that if the term is an abstraction, the result is a closure with an empty environment.

\remark*{
  That function to convert from terms to values is underspecified.
  Its domain is a superset of all terms-values; specifically, terms that contain non-constructor constants are allowed.
  All other terms yield an unspecified result.
  For the purpose of this thesis, it is sufficient to assume that \holconst{is\_value} is the domain of \holconst{sterm\_to\_value}.
}

\begin{lemma}
  For all term-values $t$, $\holconst{value\_to\_sterm}\;(\holconst{sterm\_to\_value}\;t) = t$.
\end{lemma}

\noindent
Unfortunately, the converse direction does not hold: when transforming values to terms, the distinction between recursive and non-recursive closures is lost.

\subsubsection{Syntactic predicates and relations}
\label{sec:terms:types:value:rel}

\begin{figure}[t]
  \tikzstyle{block} = [rectangle, fill=blue!10, text width=9em, text centered, node distance=2cm, minimum height=2em]
  \tikzstyle{explanation} = [rectangle, node distance=8cm, align=left, fill=green!10, minimum width=10.5cm, minimum height=3.5em, anchor=west, text width=10cm]
  \tikzstyle{line} = [draw, -latex', thick]
  \tikzstyle{explline} = [draw, densely dashed, thick, color=black!40]
  \centering
  \begin{tikzpicture}
    \node [block] (pred) {\holconst{value\_pred}};
    \node [block, below of=pred, node distance=2cm] (sterm-pred) {\holconst{value\_sterm\_pred}};

    \path [line] (sterm-pred) -- (pred);

    \node [explanation, right of=pred] (explpred) {
      $\holtypejudgement{P}{(\holtype{string}\holmap\holtype{value}) \Rightarrow (\holtype{term} \times \holtype{sterm})\;\holtype{list} \Rightarrow \holtype{bool}}$ \\
      $\holtypejudgement{Q}{\holtype{string} \Rightarrow \holtype{bool}}$ \\
      $\holtypejudgement{R}{\holtype{string}\;\holtype{set} \Rightarrow \holtype{bool}}$
    };
    \path [explline] (pred) -- (explpred);

    \node [explanation, right of=sterm-pred] (explsterm-pred) {
      $\holtypejudgement{S}{\holtype{sterm} \Rightarrow \holtype{bool}}$ \\
      $\holconst{pred}\;v \implies S\;(\holconst{value\_to\_sterm}\;v)$
    };
    \path [explline] (sterm-pred) -- (explsterm-pred);

    \node [above of=pred, node distance=1.2cm] {\bfseries Locale};
    \node [above of=explpred, node distance=1.2cm] {\bfseries Parameters \& Assumptions};
  \end{tikzpicture}
  {\small\tikz[baseline=-.3em] \path[line] (0, 0) -- (1, 0); sublocale}
  \begin{flalign*}
    \text{where:} &&
    \holconst{pred}\;(\holconst{Vconstr}\;\mathit{name}\;\mathit{vs}) &= Q\;\mathit{name} \wedge (\forall v \in \mathit{vs}.\;\holconst{pred}\;v) \\
    && \holconst{pred}\;(\holconst{Vabs}\;\cs\;\Gamma) &= P\;\Gamma\;\cs \wedge (\forall v \in \holconst{ran}\;\Gamma.\; \holconst{pred}\;v) \\
    && \holconst{pred}\;(\holconst{Vrecabs}\;\css\;\mathit{name}\;\Gamma) &= (\forall v \in \holconst{ran}\;\Gamma.\; \holconst{pred}\;v) \wedge \mathit{name} \in \holconst{dom}\;\css \wedge \phantom{foo} \\
    &&& \qquad R\;(\holconst{dom}\;\css) \wedge (\forall \mathit{name} \in \holconst{dom}\;\css.\; P\;\Gamma\;(\Gamma\;\mathit{name}))
  \end{flalign*}

  \caption{Syntactic locales on values (predicates)}
  \label{fig:terms:types:value:pred}
\end{figure}

\begin{figure}[t]
  \tikzstyle{block} = [rectangle, fill=blue!10, text width=9em, text centered, node distance=2cm, minimum height=2em]
  \tikzstyle{explanation} = [rectangle, node distance=8cm, align=left, fill=green!10, minimum width=10.5cm, minimum height=3.5em, anchor=west, text width=10cm]
  \tikzstyle{line} = [draw, -latex', thick]
  \tikzstyle{explline} = [draw, densely dashed, thick, color=black!40]
  {
  \centering
  \begin{tikzpicture}
    \node [block] (rel) {\holconst{value\_struct\_rel}};

    \node [explanation, right of=rel] (explrel) {
      $\holtypejudgement{Q}{\holtype{value} \Rightarrow \holtype{value} \Rightarrow \holtype{bool}}$ \\
      $Q\;t_1\;t_2 \implies t_1 \simeq t_2$ \\
      $\holconst{Vconstr}\;\mathit{name}_1\;\mathit{ts}_1 \simeq \holconst{Vconstr}\;\mathit{name}_2\;\mathit{ts}_2 \biimplies$ \\
      \quad$\mathit{name}_1 = \mathit{name}_2 \wedge \holconst{rel}\;Q\;\mathit{ts}_1\;\mathit{ts}_2$
    };
    \path [explline] (rel) -- (explrel);

    \node [above of=rel, node distance=1.6cm] {\bfseries Locale};
    \node [above of=explrel, node distance=1.6cm] {\bfseries Parameters \& Assumptions};
  \end{tikzpicture}
  }

  where:
  \[
    \inferrule*{
    }{\holconst{Vabs}\;\cs_1\;\Gamma_1 \simeq \holconst{Vabs}\;\cs_2\;\Gamma_2} \quad
    \inferrule*{
    }{\holconst{Vrecabs}\;\css_1\;\mathit{name}_1\;\Gamma_1\simeq\holconst{Vrecabs}\;\css_2\;\mathit{name}_2\;\Gamma_2}
  \]
  \[
    \inferrule*{\holconst{rel}\;(\simeq)\;\mathit{ts}\;\mathit{us}
    }{\holconst{Vconstr}\;\mathit{name}\;\mathit{ts} \simeq \holconst{Vconstr}\;\mathit{name}\;\mathit{us}}
  \]

  \caption{Syntactic locale on values (relation)}
  \label{fig:terms:types:value:rel}
\end{figure}

§\ref{sec:terms:algebra:thy:preds} describes locales providing syntactic predicates on term types.
Unfortunately, the \holtype{value} type cannot be made an instance of the \holclass{term} class, because it has no notion of free variables: values are always thought to be closed.

Nonetheless, just like for terms, there needs to be some predicate that can be used to enforce syntactic properties on values, because the clauses in a closure may be of arbitrary shape.
\cref{fig:terms:types:value:pred} lists two locales:
\holconst{value\_pred} introduces a generic predicate \holconst{pred} for wellformedness checks (§\ref{sec:intermediate:wellformed}) on values and \holconst{value\_sterm\_pred} establishes a connection to predicates on \holtype{sterm}s.
Contrary to the locales on terms, \holconst{value\_pred} is used to construct a predicate of type $\holtype{value} \Rightarrow \holtype{bool}$, instead of providing more results about existing predicates.
The motivation behind this is that recursion on values -- because of their complicated structure -- is cumbersome and repetitive, especially termination proofs.
This locale avoids such duplication.

The locale parameters and the definition of \holconst{pred} itself is sufficiently technical to warrant a high-level explanation.
$P$ checks a list of clauses against an environment.
This can, for example, be used to ensure that all free variables in the clauses are bound in the environment.
Consequently, the instantiation
%
\begin{align*}
  P\;\Gamma\;\cs &= (\forall (p, t)\in\cs.\;\holconst{frees}\;t \subseteq (\holconst{dom}\;\Gamma \cup \holconst{frees}\; p)) \\
  Q\;\mathit{name} &= \holconst{True} \\
  R\;\mathit{names} &= \holconst{True}
\end{align*}
%
yields a predicate that is true if and only if a value is closed.

$Q$ and $R$ check the names of the constructor and the closures.
This is used to avoid collisions with other kinds of constants.
In the compiler, this becomes relevant at multiple stages; for example, to ensure that no constant with a definition is also, say, a data constructor (see also §\ref{sec:deep} for details).

\remark*{
  The \holconst{pred} meta-predicate always checks, regardless of the concrete instantiation, that in a recursive closure, the active function is present.
  This has merely been done for convenience.
}

\noindent
Continuing with the above example, the second locale \holconst{value\_sterm\_pred} can be instantiated by defining $S = \holconst{closed}$.
It remains to be proved that $\holconst{pred}\;v$ implies $\holconst{closed}\;(\holconst{value\_to\_sterm}\allowbreak\;v)$, which is mostly technical and uninteresting.

\cref{fig:terms:types:value:rel} introduces a locale for structural relations between values.
The $\simeq$ relation represents a basic structural equality that only considers the shape of the value; and in the case of constructors, recursively.
%
\begin{lemma}
  If $t \simeq u$ and $t$ or $u$ contain no closures, then $t = u$.
\end{lemma}
%
\begin{corollary}\label{thm:terms:types:value:rel:collapse}
  If $Q\;t\;u$ and $t$ or $u$ contain no closures, then $t = u$.
\end{corollary}
%
\begin{corollary}
  $\simeq$ itself is a structural value relation.
\end{corollary}

\noindent
A predicate that ensures the absence of closures can be easily obtained by instantiating \holconst{pred} as follows:
%
\begin{align*}
  P\;\Gamma\;\cs &= \holconst{False} \\
  Q\;\mathit{name} &= \holconst{True} \\
  R\;\mathit{names} &= \holconst{False}
\end{align*}
%
All three locales provide further results about matching, but this requires additional material on patterns.
I will revisit this in §\ref{sec:terms:types:pat}.

\subsubsection{Extensional equivalence}
\label{sec:terms:types:value:ext}

\begin{code}[t]
  \[
    \inferrule*{
      \holconst{rel}\;(\approx_\text{e})\;\mathit{vs}\;\mathit{us}
    }{
      \corre{\holconst{Vconstr}\;\mathit{name}\;\mathit{vs}}{\holconst{Vconstr}\;\mathit{name}\;\mathit{us}}
    }
  \]
  \[
    \inferrule*{
      \forall x \in \holconst{ids}\;(\Lambda\;\cs).\; \corre{\sigma_1\;x}{\sigma_2\;x}
    }{
      \corre{\holconst{Vabs}\;\cs\;\sigma_1}{\holconst{Vabs}\;\cs\;\sigma_2}
    }
    \qquad
    \inferrule*{
      \forall \cs \in \holconst{range}\;\css.\; \forall x \in \holconst{ids}\;(\Lambda\;\cs).\; \corre{\sigma_1\;x}{\sigma_2\;x}
    }{
      \corrv{\holconst{Vrecabs}\;\css\;\mathit{name}\;\sigma_1}{\holconst{Vrecabs}\;\css\;\mathit{name}\;\sigma_2}
    }
  \]
  \caption{Extensional equivalence of \holtype{value}s}
  \label{code:terms:types:value:ext}
\end{code}

A special kind of structural relation on terms is extensionality (\cref{code:terms:types:value:ext}).
It compares the captured environments based on the \emph{identifiers}\emph{identifier} that occur in the body of the closure.
The set of identifiers comprises the sets of free variables and constants: $\holconst{ids}\;t = \holconst{frees}\;t \cup \holconst{consts}\;t$.\holconstindex{ids}

As outlined earlier in this section, the closures cases of the \holtype{value} type are thought to contain environments mapping variables to values.
This is only true up until a particular phase in the compiler (§\ref{sec:intermediate:ml}).
From that part on, the environments contain mappings for both variables and constants.
This explains why the extensional equivalence relation uses the set of identifiers to compare the environments.

\begin{lemma}
  Extensional equivalence is reflexive and a structural value relation.
\end{lemma}

\subsection{Proper patterns (\holtype{pat})}
\label{sec:terms:types:pat}
\holtypeindex{pat}

\begin{code}[t]
  \begin{lstlisting}[language=Isabelle]
datatype pat =
  Patvar string |
  Patconstr string (pat list)

fun mk_pat :: term $\Rightarrow$ pat where
mk_pat //pat// = ||case|| strip_comb //pat// ||of||
  (Const //s//, //args//) $\Rightarrow$ Patconstr //s// (map mk_pat //args//)
  (Free //s//, []) $\Rightarrow$ Patvar //s//\end{lstlisting}\holconstindex{mk\_pat}
  \caption{Proper patterns}
  \label{code:terms:types:pat}
\end{code}

The \holtype{value} type, instead of using binary function application as all other term types, uses $n$-ary constructor application.
This introduces a conceptual mismatch between (binary) patterns and values.
To make some proofs easier, and to introduce an intermediate layer between \holtype{term} patterns and CakeML patterns, I have introduced a \emph{dedicated pattern} type of $n$-ary patterns (\cref{code:terms:types:pat}).
The function $\holtypejudgement{\holconst{mk\_pat}}{\holtype{term} \Rightarrow \holtype{pat}}$ converts from binary to $n$-ary patterns.

\remark*{
  Note that the function is underspecified.
  Its domain is a superset of all linear terms; specifically, terms that are ``almost'' linear but in which the same variable occurs twice are allowed.
  All other terms yield an unspecified result.
}

\noindent
The remainder of this section is concerned with defining a matching function and establishing a correspondence to \holconst{match}; more specifically, for the \holtype{sterm} instantiation of \holconst{match}.
This works in two steps:
first, I define a function that matches a \holtype{pat} and an \holtype{sterm},
then, a function for \holtype{pat} and \holtype{value}.

Because patterns have a nested recursion structure, it becomes necessary to introduce some library functions that deal with lists:
\begin{align*}
  \holtypejudgement{\holconst{those}&}{\alpha\;\holtype{option}\;\holtype{list} \Rightarrow \alpha\;\holtype{list}\;\holtype{option}} \\
  \holtypejudgement{\holconst{map2}&}{(\alpha \Rightarrow \beta \Rightarrow \gamma) \Rightarrow \alpha\;\holtype{list} \Rightarrow \beta\;\holtype{list} \Rightarrow \gamma\;\holtype{list}}
\end{align*}
%
With them, it is now possible to define the intermediate matching function:
%
\begin{lstlisting}
fun match' :: pat $\Rightarrow$ sterm $\Rightarrow$ (string $\holmap$ sterm) option where
match' (Patvar //name//) //t// = Some [//name// $\mapsto$ //t//]
match' (Patconstr //name// //ps//) //t// = ||case|| strip_comb //t// ||of||
  (Sconst //name'//, //vs//) $\Rightarrow$
    ||if|| //name// = //name'// $\wedge$ length //ps// = length //vs// ||then||
      map (foldl ($\envadd{}{}$) []) (those (map2 match' //ps// //vs//))
    ||else||
      None
  _ $\Rightarrow$
    None
\end{lstlisting}
%
\begin{lemma}\label{thm:terms:types:pat:eq}
  For all linear patterns $p$ and term-values $t$: $\holconst{match}\;p\;t$ is equal to $\holconst{match'}\allowbreak\;(\holconst{mk\_pat}\allowbreak\;p)\allowbreak\;t$.
\end{lemma}

\noindent
The proof requires a custom induction rule on linear patterns:
%
\begin{lemma}[Linear induction]
  \[
    \inferrule*{
      \holconst{linear}\;t \\
      \forall s.\; P\;(\holconst{Free}\;s) \\
      \forall \mathit{name}\;\mathit{args}.\; \holconst{linears}\;\mathit{args} \implies (\forall \mathit{arg} \in \mathit{args}.\; P\;\mathit{arg}) \implies P\;(\listcomb{\mathit{name}}{\mathit{args}})
    }{P\;t}
  \]
\end{lemma}
%
\begin{proof}
  The proof proceeds by well-founded induction on the size measure.
  Clearly, in the recursive case, all constituent patterns are smaller than their combination.
  Because $t$ is linear, the cases are exhaustive.
\end{proof}

\noindent
Finally, the desired matching function, together with its correctness statement, can be defined:
%
\begin{lstlisting}
fun vmatch :: pat $\Rightarrow$ value $\Rightarrow$ (string $\holmap$ value) option where
vmatch (Patvar //name//) //v// = Some [//name// $\mapsto$ //v//]
vmatch (Patconstr //name// //ps//) (Vconstr //name'// //vs//) =
  ||if|| //name// = //name'// $\wedge$ length //ps// = length //vs// ||then||
    map (foldl op $\envadd{}{}$ []) (those (map2 vmatch //ps// //vs//))
  ||else||
    None
vmatch _ _ = None
\end{lstlisting}\holconstindex{vmatch}
%
\begin{lemma}[$n$-ary vs. binary patterns]
  For all linear patterns, the result of \holconst{vmatch} corresponds (with respect to \holconst{value\_to\_sterm}) to \holconst{match}. Formally:
  \begin{align*}
    \holconst{linear}\;p \implies &\holconst{rel}\;(\holconst{rel}\;(\lambda v\; t.\; t = \holconst{value\_to\_sterm}\;v))\;\sigma_1\;\sigma_2 \\
    \text{where } &\sigma_1 = \holconst{vmatch}\;(\holconst{mk\_pat}\;p)\;v \\
    &\sigma_2 = \holconst{match}\;p\;( \holconst{value\_to\_sterm}\;v)
  \end{align*}
\end{lemma}

\noindent
The proof first establishes a correspondence between \holconst{match'} and \holconst{vmatch} and then composes the result with \cref{thm:terms:types:pat:eq}.

Having introduced proper patterns and matching of values with patterns, \cref{thm:terms:algebra:preds:match} can be adapted accordingly in the context of the predicate locale (\cref{fig:terms:types:value:pred}):
%
\begin{lemma}
  If a value $v$ for which $\holconst{pred}\;v$ holds matches a pattern with resulting environment $\sigma$, then \holconst{pred} holds for the range of $\sigma$.
\end{lemma}

\noindent
Similarly to §\ref{sec:terms:algebra:thy:preds}, the relation locale (\cref{fig:terms:types:value:rel}) admits relational results on matching.
For example, the equivalent to \cref{thm:terms:algebra:match-rel} is:
%
\begin{lemma}
  For all structural value relations $Q$:
  If $Q\;t_1\;t_2$, then
  \[ \holconst{rel}\;\allowbreak(\holconst{rel}\;Q)\allowbreak\;(\holconst{vmatch}\;p\;t_1)\;(\holconst{vmatch}\;p\;t_2) \]
\end{lemma}