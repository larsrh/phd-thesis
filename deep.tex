% !TeX spellcheck = en_GB
% !TeX encoding = UTF-8

\noindent
The \keyword{deep embedding} phase happens after preprocessing (§\ref{sec:preproc}) and is special:
whereas preprocessing stays entirely within the realm of ML to alter (or derive new) HOL definitions and the compiler (§\ref{sec:compiler}) stays entirely within the realm of term languages modelled in Isabelle, deep embedding lifts definitions into a model.
It bridges the gap between the ML and the HOL world.

Starting with a HOL definition, this phase constructs a \emph{reified}\index{reification} definition in the deeply embedded term language given in \cref{code:background:rewriting:term}.
This term language corresponds closely to the \texttt{term} data type of Isabelle's implementation (using de~Bruijn indices \cite{debruijn1972lambda}), but without types and schematic variables.
To establish a formal connection between the original and the reified definitions, a ``family of relations'' is used, ``defined by induction on types'' \cite{tait1967interpretations}.
This concept of a \emph{logical relation} is well-understood in literature \cite{hermida2014logical,tait1967interpretations} and can be nicely implemented in HOL using type classes.

This compiler phase can be structured into three parts:
the mapping -- implemented in ML -- between raw Isabelle terms and deeply-embedded \holtype{term}s (§\ref{sec:deep:operation}),
the HOL relations that certify correspondence between both representations (§\ref{sec:deep:rel}),
and the proof tactics -- also implemented in ML -- that establish theorems about concrete mappings (§\ref{sec:deep:proofs}).