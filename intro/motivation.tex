% !TeX spellcheck = en_GB
% !TeX encoding = UTF-8

\section{Motivation}
\label{sec:intro:motivation}

The purpose of using an interactive theorem prover -- or, more generally, any kind of prover -- is to provide high assurance and trust in the produced artifacts.
The range of applications is vast: program and hardware verification, automata and formal languages, security and confidentiality guarantees, analysis and probability theory, topology, and even general relativity, to name just a few~\cite{avigad2018milestones}.
The Isabelle community also runs a peer-reviewed repository of formal proof developments, the \emph{Archive of Formal Proofs}.\footnote{\url{https://www.isa-afp.org/}}

As outlined in §\ref{sec:intro:isabelle}, Isabelle is designed in such a way that proofs of statements that are accepted by the system can generally be considered to hold.
However, a proof is just a proof; in absence of other mechanisms, they ``do nothing''.
This is why the \emph{extraction} of (or equivalently, transformation into) executable code is not just an afterthought, but rather an important area of research.

While Isabelle offers a rich toolkit for functional programming and mathematical specifications, so far, there is no trustworthy translation to executable code.
This means that the guarantees provided by the system end at its boundaries.
The generated code bears -- apart from an unverified algorithm having produced it -- no connection to the original specification.
This thesis aims to bridge the gap by developing a verified compiler from Isabelle/HOL to CakeML.

The compiler operates in multiple stages that can be roughly characterized as \emph{preprocessing,} \emph{deep embedding,} and \emph{compilation phases}.
All stages are either \keyword{certifying} or \keyword{verified}, i.e., an error in the implementation of the compiler would lead to an Isabelle error that is displayed to the user.
Under no circumstances will the system emit executable code that once compiled produces an erroneous result.