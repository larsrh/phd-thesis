% !TeX spellcheck = en_GB
% !TeX encoding = UTF-8

\section{How to read this thesis}

\authorship{
  A paragraph with a pencil indicates that the following section -- or portions thereof -- has appeared previously in another publication (§\ref{sec:intro:contrib}).
  Unless stated otherwise, in publications with a coauthor, I have contributed the majority of the content, including implementation.
}

\remark{
  Particularly thorny issues, or complicated design decisions stemming from restrictions of Isabelle or CakeML, are described in a box decorated with Bourbaki's dangerous bend symbol.
  They are not crucial for understanding and can be safely skipped.
}

\noindent
When describing a formalization, I frequently refer to theories.\thyrefafp{CakeML}{Ast}
Such references (for example to the Archive of Formal Proofs) appear in the margin.
For other non-bibliographic references, for example to external tools, I use footnotes.

Some sections in this thesis consist of large quantities of definitions, lemmas, and proofs.
They have been simplified and streamlined for better presentation.
Their actual Isabelle representation can be found in the formalization.