% !TeX spellcheck = en_GB
% !TeX encoding = UTF-8

\section{Isabelle}
\label{sec:intro:isabelle}

Isabelle is a system in the category of \emph{interactive theorem provers}~\cite{nipkow2014semantics}.
These systems provide a facility for \emph{machine-checked proofs:} a user may present a proof phrased in a formal language to the system which will then check its correctness according to the logic implemented in the system.

Working with \emph{interactive} theorem provers has often been compared to playing a video game, in which the user's objective is to navigate some rules (the logics) to achieve their goal (the proof), while constantly facing adversities (flaws).
Interactivity here means that feedback is immediate and the user can change their reasoning steps incrementally, repeated until both user and machine are satisfied.
To that end, Isabelle ships with an integrated development environment providing similar amenities to those of industry-strength programming languages.

The immediate question is the trustworthiness of Isabelle itself.
As is common in the \emph{LCF} family, its implementation strategy is to clearly delineate a \emph{kernel} providing a set of primitive inference rules and bookkeeping of definitions and other logical content.
All reasoning goes through this kernel, which -- being only a small part of the full system -- can be inspected by critical readers for flaws.
The kernel is implemented in the programming language Standard ML whose strong abstraction and type soundness guarantees make it particularly suitable for proof assistants.
In fact, Standard ML's predecessor, \emph{Meta Language} (\emph{ML} for short) has been designed by Robin Milner as the implementation language for the LCF proof assistant~\cite[§6]{plotkin2000milner}.
Furthermore, there is research on verification of proof kernels of similar systems themselves~\cite{kumar2016phd,kumar2016self}.