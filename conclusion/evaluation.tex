% !TeX spellcheck = en_GB
% !TeX encoding = UTF-8

\section{Results}
\label{sec:conclusion:eval}

The result of this thesis is a substantial formalization of a real-world compiler.

\begin{center}
  \begin{tabular}{lrr}
    \textbf{Component} & \textbf{Size} (ML) & \textbf{Size} (Isar) \\
    Dictionary Construction & 2,100 & 800 \\
    Higher-Order Term Algebra & n/a & 3,800 \\
    CakeML {\small\itshape (excluding generated parts)} & 300 & 4,100 \\
    CupCakeML & n/a & 900 \\
    Preprocessing & 1,100 & 300 \\
    Compiler & n/a & 14,100 \\\hline
    & 3,500 & 24,000
  \end{tabular}
\end{center}

\noindent
The compiler is able to emit concrete syntax (using s-expressions, §\ref{sec:intermediate:cakeml}) of CakeML that can be consumed by the CakeML compiler, version 2.0.
It comes with a correctness proof that covers all phases and a completeness proof that only covers some phases.
For that reason, it is not advisable to turn off type inference when running the CakeML compiler.

A remaining issue is performance of the compiler.
In particular, the dictionary construction and the deep embedding are expensive operations.
The former needs to define new functions which are subject to the overhead of the \holcommand{function} command.
The latter carries out proofs by induction, potentially handling large terms.
Luckily, both can be sped up by using parallelism, which is easily achieved with Isabelle/ML.