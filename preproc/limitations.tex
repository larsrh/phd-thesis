% !TeX spellcheck = en_GB
% !TeX encoding = UTF-8

\section{Limitations}
\label{sec:preproc:limitations}

This section discusses the limitations of the preprocessing phase.
Here, \emph{limitation} can either mean that a particular step in preprocessing fails to work on a particular set of definitions, or that it produces a result that is less general than expected.
Most limitations can be avoided with specific workarounds.

\subsection{Specifiedness}

A particularly thorny issue is presented by functions that return other functions.
While \emph{currying} itself is a common idiom in functional programming, manipulation of partially applied functions would require a non-trivial data flow analysis.
As a synthetic example, consider the expression $\holtypejudgement{\holconst{map}\;(\holconst{div})\;\mathit{xs}}{(\holtype{nat} \Rightarrow \holtype{nat})\;\holtype{list}}$, with \holconst{div} being the division operator on natural numbers.
Clearly, the expression is fully specified for all $\mathit{xs}$, but its resulting list of functions is not:
Passing in $0$ to any of the resulting functions yields unspecified behaviour, or at run-time in a target language, throw an exception.

The underlying problem is that the congruence rule for \holconst{map} can only be used to extract side conditions when the return type of the function that is passed to it is not itself a function type.
More formally, the heuristic requires that no type variables are instantiated with a function type in order to work correctly.

A similar situation arises in practice in the commonly used \holconst{show} derivation framework by Sternagel and Thiemann~\cite{sternagel2014show}.
They employ Hughes' difference list representation of strings~\cite{hughes1986dlist}.
Luckily, all these functions are fully specified, i.e.\ the side condition is always true.

\begin{code}[t]
  \begin{subcode}{\linewidth}
    \begin{lstlisting}
datatype $\alpha$ list = Cons $\alpha$ ($\alpha$ list) | Nil

fun append :: $\alpha$ list $\Rightarrow$ $\alpha$ list $\Rightarrow$ $\alpha$ list where
append Nil //ys// = //ys//
append (Cons //x// //xs//) //ys// = Cons //x// (append //xs// //ys//)

fun Snoc :: $\alpha$ list $\Rightarrow$ $\alpha$ $\Rightarrow$ $\alpha$ list where
Snoc //xs// //x// = append //xs// [//x//]

code_datatype Nil Snoc\end{lstlisting}
    \caption{Full definition of a ``cons list''}
  \end{subcode}
  \begin{subcode}{\linewidth}
    \begin{lstlisting}[language=ML]
datatype 'a list = Snoc of 'a list * 'a | Nil;\end{lstlisting}
    \caption{Generated datatype definition in Standard ML}
  \end{subcode}
  \begin{subcode}{\linewidth}
    \begin{lstlisting}
lemma [code]:
  append //xs// Nil = //xs//
  append //xs// (Snoc //ys// //y//) = Snoc (append //xs// //ys//) //y//
(* proof *)\end{lstlisting}
    \caption{Defining equations for \holconst{append}}
  \end{subcode}

  \caption{Adapting a datatype to a different representation}
  \label{code:snoc}
\end{code}

\subsection{Patterns in function definitions}

The \holcommand{function} package, by default, allows only function definitions where the left-hand side matches on constructors.
Consider as an example treating list as ``snoc lists'' instead of ``cons lists'', i.e., a pair of \emph{init} and \emph{last} instead of \emph{head} and \emph{tail}.
A complete example is given in \cref{code:snoc}.
It first introduces the datatype for cons lists and defines the \holconst{append} and \holconst{Snoc} functions.
Then, it instructs the code generator to use \holconst{Nil} and \holconst{Snoc} in the target language representation.

However, the code generator cannot export code for the \holconst{append} function, because it is defined in terms of \holconst{Cons}, and aborts with an error message that \holconst{Cons} is not a constructor on the left-hand side of the defining equation.

To fix this, \cref{code:snoc} demonstrates how to add defining equations for \holconst{append} that are defined in terms of \holconst{Snoc}.
But now the \holcommand{function} package would not accept such a definition of \holconst{append}, because while \holconst{Snoc} is a constructor as far as the code generator is concerned, it is still not a constructor as far as the \holcommand{datatype} package is concerned.
The key problem is that both subsystems may have diverging notions of exactly which constructors a datatype is comprised of.

The workaround for this problem is that it is possible to use the \holcommand{function} package in a mode which allows to use arbitrary patterns on the left-hand side of a defining equation.
It is only a workaround because the package demands some additional proofs (exhaustiveness, well-definedness) that are tedious to do by hand and impossible to automate in general.

For that reason, any type adaptations, including data refinement \cite{haftmann2013refinement}, are not supported by this work.

\subsection{Pattern compatibility}\index{pattern compatibility}

The classic example for a set of equations with patterns that are difficult to deal with is the \holconst{diagonal} function~\cite{slind1999terminating,peytonjones1987implementation}:
%
\begin{lstlisting}[language=Isabelle]
fun diagonal :: bool $\Rightarrow$ bool $\Rightarrow$ bool $\Rightarrow$ nat where
diagonal //x// True False = 1
diagonal False //y// True = 2
diagonal True False //z// = 3
\end{lstlisting}
%
The difficulty arises from the overlapping patterns.
There is no possible renaming of variables such that this set of equations satisfies the pattern compatibility constraint.
Consequently, the preprocessing step will produce an error message.

The workaround is to instantiate the first two equations with $\{\holconst{True}, \holconst{False}\}$, which can be achieved with modest effort in Isabelle:
%
\begin{lstlisting}[language=Isabelle]
diagonal True True False = 1
diagonal False True False = 1
diagonal False True True = 2
diagonal False False True = 2
diagonal True False //z// = 3
\end{lstlisting}
%
The process could also be automated, but because it is potentially an expensive operation, the user has the choice how to deal with the situation.

\subsection{Congruence rules with constraints}

Consider a variant of the \holconst{sum\_list} function that first applies a function before summing up the values:
%
\begin{lstlisting}
fun sum_by :: ($\alpha$ $\Rightarrow$ $\beta$::{plus,zero}) $\Rightarrow$ $\alpha$ by $\Rightarrow$ $\beta$ where
sum_by _ [] = //0//
sum_by //f// (//x// # //xs//) = //f// //x// + sum_by //f// //xs//
\end{lstlisting}
%
In order to use this function in a higher-order recursion, a congruence rule needs to be proved.
The resulting rule is very similar to the congruence rule of \holconst{map}:
\[
  \inferrule*{
    \forall x.\; x \in \holconst{set}\;\mathit{xs} \implies f\;x = g\;x\\
    \mathit{xs} = \mathit{ys}
  }{\holconst{sum\_by}\;f\;\mathit{xs} = \holconst{sum\_by}\;g\;\mathit{ys}}
\]
%
Since \holconst{sum\_by} has sort constraints, the dictionary construction will introduce additional parameters.

Now suppose that another function \holconst{h} uses \holconst{sum\_by}.
Dictionary construction replaces occurrences of \holconst{sum\_by} in \holconst{h} by \holconst{sum\_by'}.
But \holconst{sum\_by'} has no congruence rule, which means that the termination proof of \holconst{h} must fail.

Besides the termination heuristic, I have also implemented a heuristic method to transfer congruence rules to the new constants.
The \holcommand{function} package only accepts unconditional congruence rules; consequently, this only works when function equipped with a congruence rule is fully specified and has no visible sort constraints.

For that reason, a function like \holconst{sum\_by} cannot be used as a higher-order recursion combinator.
Instead, it needs to be split into two parts: mapping and summing.
This can be registered in the code generator as follows:
%
\begin{lstlisting}
lemma sum_by[code_unfold]: sum_by //f// = sum_list $\circ$ sum_by //f//
(* proof *)
\end{lstlisting}

\subsection{Preservation of termination}
The following pathological example exhibits the problem that some functions cannot be proved to terminate after elimination of sort constraints:
%
\begin{lstlisting}[language=Isabelle]
function sum_set :: $\alpha$::{finite,comm_monoid_add,linorder} set $\Rightarrow$ $\alpha$ where
sum_set //S// = (if //S// = {} then $0$ else Min //S// + sum_set (//S// - {Min //S//}))
\end{lstlisting}
%
This function, analogously to \holconst{sum\_list}, should compute the sum of a set.
It can only be proved terminating because of the sort constraints:
otherwise, there may be infinitely many elements in $S$, no well-defined minimum element, or the result may be depending on the order.
With the constraint, the termination relation is the cardinality of the set which decreases with each recursive call.

However, the termination heuristics cannot cope with this example.
The dictionary construction removes all sort constraints from the type variables; instead introducing value parameters.
The dictionary type for \holclass{finite} would be isomorphic to \holtype{unit}, because the class does not have any parameters.
But now, the new termination relation cannot be simulated by the old one:
it has to deal with arbitrary, possibly infinite sets.

Fortunately, function definitions like this are rare.
If necessary, they can be replaced by a recursion on lists as follows:
%
\begin{lstlisting}[language=Isabelle]
lemma sum_set_set[code_unfold]: sum_set (set //xs//) = sum_list (remdups //xs//)
(* proof *)
\end{lstlisting}
%
This replaces all occurrences of \holconst{sum\_set} applied to a finite set of elements $\mathit{xs}$ by an application of \holconst{sum\_list} (after removing duplicate elements).
After the automatic preprocessing phase of the code generator, no traces of recursion through sets will be left.
This pattern of replacing logical recursion on sets by executable recursion on lists is common in Isabelle's standard libraries.

Furthermore, the heuristic cannot prove all function definitions that are still terminating after class elimination to be terminating.
The following circumstances prevent a termination proof to be ported:
\begin{itemize}
  \item
    Recursive calls that receive the result of another recursive call as an argument.
    A classic example is \emph{McCarthy's 91 function} \cite{manna1970properties}, which is defined as follows:
      \[ \holconst{f91}\;n = (\holkeyword{if}\; 100 < n\; \holkeyword{then}\; n - 10 \;\holkeyword{else}\; \holconst{f91}\; (\holconst{f91}\; (n + 11))) \]
    The termination proof requires an additional lemma that gives an estimate on the return value.
    Krauss' \holcommand{function} package can deal with this specification \cite[§2.7.2]{krauss2009fun}.
    My heuristic cannot, because the generated termination condition mentions the function itself.
  \item
    Functions with at least one polymorphic parameter that is not a suitable type constructor.
    The termination heuristic will ask the \holcommand{datatype} package to deliver a map function from the new relation to the old relation, which is impossible in some cases.
\end{itemize}

\noindent
When the heuristic fails, the system falls back to an automatic termination proof using the lexicographic order method.
Presently, for technical reasons, it is impossible to give a manual proof when the automatic proof also fails.
