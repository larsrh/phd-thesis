% !TeX spellcheck = en_GB
% !TeX encoding = UTF-8

\section{Constructor functions}
\label{sec:preproc:constructor}

\authorship{
  Portions of this section appear in the AFP entry \citetitle{hupel2017constructor}, authored by \citeauthor{hupel2017constructor}~\cite{hupel2017constructor}.
  Users are recommended to refer to that AFP entry for the latest documentation.
}

\noindent
For each data constructor $\holconst{C}_i$ of datatype $\tau$ (as exemplified above), Isabelle generates a function $\holtypejudgement{\holconst{C}_i}{\kappa_{i,1} \Rightarrow \ldots \Rightarrow \kappa_{i,n_i} \Rightarrow \tau}$.
It behaves like a regular function, in that it can be passed to a higher-order function, partially applied, and generally be treated as a value.

In implementations of programming languages, data constructors behave differently:
An application of $\holconst{C}_i$ to $n_i$ arguments cannot be reduced further, but instead creates an ``atomic'' value.
Hence, the question arises on how to represent partially applied constructors.

Different functional languages answer this question differently.
OCaml disallows partial application and forces users to explicitly write an anonymous function, e.g.\ $\lambda x.\; \holconst{C}_i\;t_1\;\ldots\;t_{n_i-1}\;x$, as does CakeML.
Standard ML however allows that and internally constructs such a function.
The naive approach of plain eta-expansion does not work in Isabelle, because many routines, including code preprocessing, spontaneously eta-contract terms.

To bridge the gap, preprocessing introduces synthetic \emph{constructor functions}.
For each data constructor $\holconst{C}_i$, it defines an additional function $\holconst{C'}\!\!\!_i$ as $\holconst{C'}\!\!\!_i\;t_1\;\ldots\;t_{n_i} = \holconst{C}_i\;t_1\;\ldots\;t_{n_i}$, having the same type as $\holconst{C}_i$.
Every occurrences of $\holconst{C}_i$ is replaced by $\holconst{C'}\!\!\!_i$, except in the definition of $\holconst{C'}\!\!\!_i$.
Thus, it is statically guaranteed that each constructor call is fully applied.

For example, the term $\holconst{map}\;((\#)\;1)$ is rewritten to $\holconst{map}\;((\#')\;1)$.
Unfolding the synthetic definition of $(\#')$ yields the expression $\holconst{map}\;(\lambda x.\; 1 \cons x)$.
There, $\#$ appears in fully-applied form.