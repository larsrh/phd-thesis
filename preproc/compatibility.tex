% !TeX spellcheck = en_GB
% !TeX encoding = UTF-8

\section{Pattern compatibility}
\label{sec:preproc:compatibility}

The pattern elimination phase of the compiler (§\ref{sec:intermediate:elim}) demands a stronger guarantee than non-overlapping patterns.
The reason for that is deferred until that section, whereas here, I will explain how this requirement is ensured during preprocessing.

\begin{definition}[Overlapping patterns]\label{def:preproc:compatibility:overlapping}\index{pattern overlap}
  Two patterns $t_1$ and $t_2$ \emph{overlap} if there is a term $u$ that is matched by both $t_1$ and $t_2$.
\end{definition}

\remark*{
  The definition for overlapping patterns given above differs from the one used in the formalization.
  A refined version based on different primitives is presented in §\ref{sec:terms:algebra:thy:subst}.
}

\begin{definition}[Pattern compatibility]\index{pattern compatibility}\label{def:preproc:compatibility}
  Two patterns $t$ and $u$ are compatible:
  \begin{itemize}
    \item
      if both are applications, i.e., there are $t_1$, $t_2$, $u_1$, and $u_2$ such that $t = t_1\;t_2$ and $u = u_1\;u_2$, then
      \begin{enumerate}
        \item $t_1$ and $u_1$ must be compatible, and
        \item if $t_1 = u_1$, then $t_2$ and $u_2$ must be compatible,
      \end{enumerate}
    \item
      otherwise,
      \begin{enumerate}
        \item $t = u$, or
        \item $t$ and $u$ do not overlap.
      \end{enumerate}
  \end{itemize}
\end{definition}

\noindent
For example, the following function definition does not satisfy pattern compatibility, because the first parameter is named differently in the two equations:
%
\begin{lstlisting}[language=Isabelle]
fun map where
map //f// [] = []
map //g// (//x// # //xs//) = (* ... *)
\end{lstlisting}

\begin{lemma}[Reflexivity]
  $t$ is compatible to itself.
\end{lemma}

\begin{lemma}\label{thm:preproc:compatibility:cases}
  For two compatible patterns $t$ and $u$, they are either equal or do not overlap.
\end{lemma}

\begin{proof}
  I first prove the weaker result that $t$ and $u$ are equal or do not overlap by induction on the definition of pattern compatibility.
  The actual lemma follows together with reflexivity.

  Consider the case where both $t$ and $u$ are not applications.
  Then, by definition, $t = u$ or $t$ and $u$ do not overlap.

  The remaining case is where both are applications, i.e., $t = t_1\;t_2$ and $u = u_1\;u_2$.
  By definition, $t_1$ and $u_1$ must be compatible.
  I perform a case distinction on $t_1 = u_1 \wedge u_1 = u_2$.
  \begin{itemize}
    \item
      If $t_1 = u_1 \wedge u_1 = u_2$, then $t = u$.
    \item
      If $t_1 = u_1$, then $t_2 \neq u_2$.
      By definition, $t_2$ and $u_2$ are also compatible.
      Applying the induction hypothesis yields that $t_2 = u_2$ (contradiction) or $t_2$ and $u_2$ do not overlap.
      Hence, $t_1\;t_2$ and $u_1\;u_2$ do not overlap.
    \item
      Otherwise, I know that $t_1 \neq u_1$.
      Applying the induction hypothesis on the fact that $t_1$ and $u_1$ are compatible yields that $t_1 = u_1$ (contradiction) or $t_1$ and $u_1$ do not overlap.
      Hence, $t_1\;t_2$ and $u_1\;u_2$ do not overlap. \qedhere
  \end{itemize}
\end{proof}

\noindent
The preprocessing will rename $g$ to $f$ to ensure that matching sub-patterns are also equal.
This does not work in all cases (§\ref{sec:preproc:limitations}).

The algorithm itself is rather simple:
starting out with an empty set of defining equations, each defining equation gets added to the set.
Before adding, it is compared with all existing equations in the set and variables are renamed if necessary and possible.
The resulting set are the new defining equations.

A more formal definition of pattern compatibility will be given in §\ref{sec:terms:algebra:thy:matching}.