% !TeX spellcheck = en_GB
% !TeX encoding = UTF-8

\section{Embedding operation}
\label{sec:deep:operation}

The embedding operations lifts HOL definitions into the \holtype{term} type.
I use angle brackets to denote an embedded term: \embed{t}\index{$\langle\rangle$@$\langle\rangle$ (symbol)}, where $t$ is an arbitrary HOL expression of any type and the result \embed{t} is a HOL value of type \holtype{term}.
It is a purely syntactic transformation, without preliminary evaluation or reduction, and discards type information.
The following examples illustrate the operation and its typographical conventions:
%
\begin{align*}
  \embed{x} &= \holconst{Free}\;\texttt{"x"} \\
  \embed{\holconst{f}} &= \holconst{Const}\;\texttt{"f"} \\
  \embed{x\;\#\;\mathit{xs}} &= \app{\app{\holconst{Const}\;\texttt{"List.list.Cons"}}{\embed x}}{\embed{\mathit{xs}}} \\
  \embed{\lambda x y.\; \holconst{f}\;y\;x} &= \Lambda\;(\Lambda\;(\app{\app{\embed{\holconst{f}}}{\holconst{Bound}\; 0}}{\holconst{Bound}\;1}))
\end{align*}
%
\noindent
Contrary to the formalization, I will use the \embed{t} notation in a more flexible way and allow it to also represent values of the other term types (§\ref{sec:terms:types}), e.g.\ \holconst{pterm}.

The correspondence between an embedded term $t$ to a HOL term $a$ of type $\tau$ is expressed with the syntax $\evalr\R t a$.
The relation can be described as ``using the rule set \R, the term $t$ can be rewritten to $\embed{a}$''.
Most importantly, this means the correspondence relation is aware of the term-rewriting semantics of \holtype{term} (\cref{code:background:rewriting:semantics}), where \R\ is the set of all known defining equations.
Furthermore, the notation $\evalr\R t a$ actually represents a family of relations.
Abstractly, it is defined in the \holclass{embed} type class.

Before the relations are discussed in detail in §\ref{sec:deep:rel}, I will show an example of the embedding operation and explain its implementation.

\paragraph{Example}
Consider the \holconst{map} function on lists:
\[
  \begin{array}{lllcl}
    \holconst{map}&f&[] &=& [] \\
    \holconst{map}&f&(x \cons \mathit{xs}) &=& f\;x \cons \holconst{map}\;f\;\mathit{xs}
  \end{array}
\]

\noindent
The result of embedding this function is a set of rules \holconst{map'}.
If written directly by the user, it would look as follows:
\begin{lstlisting}[basicstyle=\footnotesize\ttfamily,keywordstyle=\sffamily\bfseries\small\color{Blue}]
definition map' :: (term $\times$ term) set where
map' =
  {(Const "List.list.map" $\$$ Free "f" $\$$ (Const "List.list.Cons" $\$$ Free "x" $\$$ Free "xs"),
      Const "List.list.Cons" $\$$ (Free "f" $\$$ Free "x") $\$$ $\ldots$),
   (Const "List.list.map" $\$$ Free "f" $\$$ Const "List.list.Nil",
      Const "List.list.Nil")}
\end{lstlisting}
%
Additionally, the theorem $\evalr{\holconst{map'}}{\holconst{Const}\;\texttt{"List.list.map"}}{\holconst{map}}$ is proved using a custom tactic (§\ref{sec:deep:proofs}).
Constant names like \holconst{List.list.map} come from the fully qualified internal names in Isabelle.

\begin{code}
  \begin{lstlisting}[mathescape=false,language=ML]
fun embed schematic t =
  let
    fun
      aux (Const (n, _)) = @{term Const} $ HOLogic.mk_string n
    | aux (Free (n, _)) =
        if schematic then
          error "free variables are not supported"
        else
          @{term Free} $ HOLogic.mk_string n
    | aux (Bound i) = @{term Bound} $ HOLogic.mk_number @{typ nat} i
    | aux (t $ u) = @{term App} $ aux t $ aux u
    | aux (Abs (_, _, t)) = @{term Abs} $ aux t
    | aux (Var ((n, i), _)) =
        if schematic then
          @{term Free} $ HOLogic.mk_string (n ^ "." ^ Value.print_int i)
        else
          error "schematic variables are not supported"
  in aux t end
  \end{lstlisting}

  \caption{Full ML implementation of the embedding operation on terms}
  \label{code:deep:embed}
\end{code}

\paragraph{Implementation}
\cref{code:deep:embed} shows the Isabelle/ML implementation of the embedding operation on terms.
It lifts the Pure term constructors to their equivalents in the \holconst{term} type.

Note that the \holconst{term} type only knows free variables, but no schematic variables.
Depending on the context where this function is used, either schematic or free variables are mapped to free variables in \holconst{term}.

This function can then be subsequently used to lift sets of equations.
As illustrated by the example, the resulting type is $(\holtype{term} \times \holtype{term})\;\holtype{set}$: equations are split into left- and right-hand side (\cref{def:terms:algebra:thy:eq}).
The set contains all transitive dependencies according to the code graph (§\ref{sec:preproc:dict:elim:impl}).

Because the set of embedded defining equations is now subject to a term-rewriting semantics, I will refer to it as a rule set\index{rule set} (§\ref{sec:terms:algebra:thy:matching}).

\begin{code}[t]
  \begin{lstlisting}
type_synonym c_info = nat $\times$ string

locale constructors =
  fixes C_info :: string $\holmap$ c_info
begin

definition C :: string set where
C = dom //C_info//

end

locale constants = constructors +
  fixes heads :: string set
  assumes heads $\cap$ C = $\emptyset$
begin

definition all_consts :: string set where
all_consts = heads $\cup$ C

end\end{lstlisting}\holconstindex{C}\holconstindex{C\_info}\holconstindex{heads}\holconstindex{all\_consts}
  \caption{Definitions of the \holconst{constructors} and \holconst{constants} locales}
  \label{code:deep:constructors}
\end{code}

Observe that both constructor names (e.g.\ \holconst{Cons}) and function names (e.g.\ \holconst{map}) look the same in this representation.
But just like in the dictionary construction (§\ref{sec:preproc:dict}), they require different treatment in some phases of the compiler.
For that reason, the deep embedding also produces a set of constructor names, which is a subset of all constants that occur.
This is captured in the \holconst{constructors} locale (\cref{code:deep:constructors}).
The constructor information consists of name, arity, and name of the type that it belongs to.

Additionally, there is also a locale that fixes a rule set and assumes some wellformedness conditions (§\ref{sec:intermediate:wellformed}).
It is a sublocale of \holconst{constants}, which ensures that there are no constants that are both constructors and head of a defining equation.
More broadly speaking, the embedding operation does not just define a set of equations, but declares an interpretation of the locale and proves all conditions.
Ill-formed rule sets would be flagged at this point.

\remark*{
  Deep embedding -- being a purely syntactic operation -- does not respect \emph{referential transparency} \cite{sep2017quotation,quine1960word}.
  In particular, constants' names become part of the embedded definition.
  As an example, consider the two functions $\holconst{id}_1\;x = x$ and $\holconst{id}_2\;x = x$.
  Both can be proved to be equal according to HOL's extensional function equality.
  However, their deep embeddings differ from each other.
  For practical purposes, this bears little relevance, because users of the compiler would not routinely alter either the embedding or the generated theorems.
}
