% !TeX spellcheck = en_GB
% !TeX encoding = UTF-8

\section{Related work}

Fallenstein and Kumar~\cite{fallenstein2015reflection} have presented a model of HOL inside HOL including a reflection proof principle.
The principle states that for any proposition $\varphi$, if the syntactic encoding of $\varphi$ is provable, then $\varphi$ holds.
This covers the entirety of of higher-order logic and requires the existence of a large cardinal.
The work has been carried out in HOL4.
In this thesis, I restrict myself to only the term-rewriting fragment of higher-order logic, meaning that the full generality of the reflection principle is not required.

Kunčar and Popescu~\cite{kuncar2018consistent,kuncar2018types} introduce a mechanism to turn statements on types into statements on sets, exploiting the common set-theory-based semantics of higher-order logic.
They perform a low-level dictionary construction using an extension of the logic.
In this work, no extension is required; instead, derived constants are re-defined through the \holcommand{function} package (§\ref{sec:preproc:dict}).