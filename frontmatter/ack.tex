% !TeX spellcheck = en_GB
% !TeX encoding = UTF-8

\section*{Acknowledgements}

{

\setlength\parindent{0pt}
\setlength{\parskip}{10pt}

Firstly, I would like to thank my advisor Tobias Nipkow.
His guidance and the close collaboration was invaluable.
I had not even finished my Master's thesis when Tobias suggested to work on this topic as a \emph{Doktorand}.
I thoroughly enjoyed my time in his research group, including the freedoms, the archive, the teaching, and the colleagues.
I had the pleasure of working with Mohammad Abdulaziz, Jasmin Blanchette, Julian Brunner, Manuel Eberl, Florian Haftmann, Maximilian Haslbeck (the elder), Johannes Hölzl, Fabian Immler, Ondřej Kunčar, Peter Lammich, Lars Noschinski, Andrei Popescu, Dimitriy Traytel, Simon Wimmer and Bohua Zhan.

I am grateful to the numerous people who have read and commented on early versions of this thesis.
They are, in alphabetical order:
Mohammad Abdulaziz, Jasmin Blanchette, Adelbert Chang, Manuel Eberl, Peter Lammich, Maximilian Haslbeck (the elder), Marek Kubica, Magnus Myreen, Tobias Pflug, and George Pirlea.
This also applies to the anonymous reviewers at POPL, ESOP, and Fundamenta Informaticae.
Additionally, I had long and fruitful discussions about some of the subjects with Konrad Slind and Freek Wiedijk.

I would also like to thank everyone involved in the CakeML project with whom I had the pleasure to collaborate with, both in person at Chalmers and remotely.
Those are, in no particular order, Ramana Kumar, Magnus Myreen, Scott Owens, Yong Kiam Tan, Johannes Åman Pohjola, Oskar Abrahamsson, Alejandro Gómez Londoño, Andreas Lööw, Michael Sproul, and Michael Norrish.
Thanks to Magnus for hosting me twice in Chalmers and encouraging me to continue my work in this direction.

My intern for two months, Yu Zhang, has contributed some of the proofs that are relevant for the CakeML semantics in Isabelle and the compiler.
That was some great work!

Last but not least, my parents always encouraged me to aim high.
Their support was what made my entire academic path possible, starting back in high school.

}

\clearpage

\thispagestyle{empty}
\phantom{ }
