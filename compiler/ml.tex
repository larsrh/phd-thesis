% !TeX spellcheck = en_GB
% !TeX encoding = UTF-8

\section{Evaluation with recursive closures}
\label{sec:intermediate:ml}

\begin{code}[t]
  \[
    \inferrule*[left=Const]{\mathit{name} \notin \C \\ \sigma\;\mathit{name} = \holconst{Some}\;v}{\seval*{\sigma}{\holconst{Sconst}\;\mathit{name}}{v}}
  \]
  \[
    \inferrule*[left=Var]{\sigma\;\mathit{name} = \holconst{Some}\;v}{\seval*{\sigma}{\holconst{Svar}\;\mathit{name}}{v}} \quad
    \inferrule*[left=Abs]{ }{\seval*\sigma{\Lambda\;\cs}{\holconst{Vabs}\;\cs\;\sigma}}
  \]
  \[
    \inferrule*[left=Comb]{
      \seval*\sigma{t}{\holconst{Vabs}\;\cs\;\sigma'} \\
      \seval*\sigma{u}{v} \\
      \holconst{find\_match}\;\cs\;v = \holconst{Some}\;(\sigma'', \mathit{rhs}) \\
      \seval*{\envadd{\sigma'}{\sigma''}}{\mathit{rhs}}{v'}
    }{\seval*\sigma{\app t u}{v'}}
  \]
  \[
    \inferrule*[left=RecComb]{
      \seval*\sigma{t}{\holconst{Vrecabs}\;\css\;\mathit{name}\;\sigma'} \\
      \css\;\mathit{name} = \holconst{Some}\;\cs \\
      \seval*\sigma{u}{v} \\
      \holconst{find\_match}\;\cs\;v = \holconst{Some}\;(\sigma'', \mathit{rhs}) \\
      \seval*{\envadd{\sigma'}{\envadd{\holconst{mk\_rec\_env}\;\mathit{css}\;\sigma'}{\sigma''}}}{\mathit{rhs}}{v'}
    }{\seval*\sigma{\app t u}{v'}}
  \]
  \[
    \inferrule*[left=Constr]{
      \mathit{name} \in \C \\
      \seval*\sigma{t_1}{v_1} \\
      \cdots \\
      \seval*\sigma{t_n}{v_n}
    }{\seval*\sigma{\holconst{Sconst}\;\mathit{name}\mathbin\$t_1\mathbin\$\ldots\mathbin\$t_n}{\holconst{Vconstr}\;\mathit{name}\;[v_1, \ldots, v_n]}}
  \]

  \[
    \holconst{mk\_rec\_env}\;\mathit{css}\;\sigma = \holconst{map\_of}\;\{ \holconst{Vrecabs}\;\mathit{css}\;\mathit{name}\;\sigma \mid (\mathit{name}, \_) \in \mathit{css} \}
  \]\holconstindex{mk\_rec\_env}
  \caption{ML-style evaluation semantics}
  \label{code:intermediate:ml}
\end{code}

CakeML distinguishes between non-recursive and recursive closures \cite{myreen2014translation}.
This distinction is also present in the \holtype{value} type.
In this step, I will close that gap by conflating variables with constants.
This necessitates a special treatment of recursive closures.
Therefore I introduce a new predicate $\seval*\sigma t v$ (\cref{code:intermediate:ml}) that -- in contrast to the previous semantics -- only has one environment.

As usual, I will explain the rules that have changed significantly from the previous semantics here:
\begin{semantics}
  \item[Const/Var]
    Constant definition and variable values are both retrieved from the same environment $\sigma$.
    I have chosen to keep the distinction between constants and variables in the \holtype{sterm} type to avoid the introduction of another intermediate term type.
    This gap will be closed in the final phase (§\ref{sec:intermediate:cakeml}).
    For technical reasons, the \textsc{Const} case here now requires an additional check that it is a non-constructor constant, which was not required in the previous semantics.
  \item[Abs]
    Identical to the previous evaluation semantics.
    Evaluation never creates recursive closures at run-time.
    Anonymous functions, e.g.\ in the term $\embed{\holconst{map}\;(\lambda x.\;x)}$, are evaluated to non-recursive closures.
    Combined with the fact that in my source language, the set of definitions is fixed before running the compiler, it follows that the evaluation semantics never needs to construct a hitherto unseen \holconst{Vrecabs} value: these are constructed by the compiler (§\ref{sec:intermediate:value:sublocale}).
  \item[RecComb]
    Almost identical to the evaluation semantics.
    Additionally, for each function $(\mathit{name},\mathit{cs}) \in \css$, a new recursive closure $\holconst{Vrecabs}\;\css\;\mathit{name}\;\sigma'$ is constructed and inserted into the environment.
    Observe that the $\mathit{cs}$ is ignored during construction, because it is contained in $\css$ itself.
    This ensures that after the first call to a recursive function, the function itself is present in the environment to be called recursively, without having to introduce coinductive environments.
    Still, some proofs require coinductive reasoning at a later point (\cref{thm:intermediate:ml}).
    The strategy itself is similar to how it is implemented in the CakeML semantics.
\end{semantics}

\noindent
By merging the rule set $\rs$ with the variable environment $\sigma$, it becomes necessary to discuss possible clashes.
Previously, the syntactic distinction between \holconst{Svar} and \holconst{Sconst} meant that \embed{x} (the syntactic variable ``x'') and \embed{\holconst{x}} (the syntactic constant ``x'') are not ambiguous:
all semantics up to the evaluation semantics clearly specify where to look for the substitute.
This is not the case in functional languages where definitions and variables are not distinguished syntactically.

Instead, I rely on the fact that the initial rule set only defines constants.
All variables are introduced by matching before $\beta$-reduction (that is, in the \textsc{Comb} and \textsc{RecComb} rules).
The \textsc{Abs} rule does not change the environment.
Hence it suffices to assume that variables in patterns must not overlap with constant names, which is guaranteed by a side-condition (§\ref{sec:intermediate:wellformed}).

The following lemma shows the direct correspondence between $\rs_\text{v}$ (§\ref{sec:intermediate:value:sublocale}) and $\holconst{mk\_rec\_env}$ (\cref{code:intermediate:ml}).
%
\begin{lemma}\label{thm:intermediate:value:rs-mk-rec-env}
  $\holconst{map\_of}\;\rs_\text{v} = \holconst{mk\_rec\_env}\;\holconst{global\_css}\;[]$
\end{lemma}
%
\noindent
The proof is technical and requires no further assumptions about the rule set.

\begin{lemma}[Extensionality]\label{thm:intermediate:ml:ext}
  Evaluation is invariant under extensional equivalence (§\ref{sec:terms:types:value:ext}).
\end{lemma}

\paragraph{Example}
Reusing the example from §\ref{sec:intermediate:value:sublocale}, I will demonstrate the evaluation of the term $\embed{\holconst{odd}\;(\holconst{Suc}\;0)}$.
Since it is a closed term, there are no free variables that need to have a value.
Let $\sigma = \rs_\text{v}$ from that example.

\[
  \inferrule*[Left=RecComb]{
    \cdots \\
    \inferrule*[left=\textdagger]{
      \inferrule*[left=RecComb]{
        \cdots \\
        \inferrule*[left=Var]{ }{
          \seval*{\envadd{\sigma}{[n \mapsto \embed{0}]}}{\embed n}{\embed 0}
        }
      }
      {\seval*{\envadd{\sigma}{[n \mapsto \embed{0}]}}{\embed{\holconst{even}\;n}}{\embed{\holconst{True}}}}
    }
    {\seval*{\envadd{[]}{\envadd{\holconst{mk\_rec\_env}\;\css\;[]}{[n \mapsto \embed{0}]}}}{\embed{\holconst{even}\;n}}{\embed{\holconst{True}}}}
  }
  {\seval*\sigma{\embed{\holconst{odd}\;(\holconst{Suc}\;0)}}{\embed{\holconst{True}}}}
\]

\noindent
The tree is annotated with the rules from the semantics.
The step \textdagger{} is not a rule application, but a simplification, because $\holconst{mk\_rec\_env}\;\css\;[] = \sigma$.
This example illustrates that \holconst{mk\_rec\_env} allows recursive calls without introducing cyclic environments.

\subsection{Correspondence relation}

\begin{code}[t]
  \[
    \inferrule*{
      \holconst{rel}\;(\approx_\text{v})\;\mathit{vs}\;\mathit{us}
    }{
      \corrv{\holconst{Vconstr}\;\mathit{name}\;\mathit{vs}}{\holconst{Vconstr}\;\mathit{name}\;\mathit{us}}
    }
  \]
  \[
    \inferrule*{
      \forall x \in \holconst{frees}\;(\Lambda\;\cs).\; \corrv{\sigma_1\;x}{\sigma_2\;x} \\
      \forall x \in \holconst{consts}\;(\Lambda\;\cs).\; \corrv{\rs\;x}{\sigma_2\;x}
    }{
      \corrv{\holconst{Vabs}\;\cs\;\sigma_1}{\holconst{Vabs}\;\cs\;\sigma_2}
    }
  \]
  \[
    \inferrule*{
      \forall \cs \in \holconst{range}\;\css.\; \forall x \in \holconst{frees}\;(\Lambda\;\cs).\; \corrv{\sigma_1\;x}{\sigma_2\;x} \\
      \forall \cs \in \holconst{range}\;\css.\; \forall x \in \holconst{consts}\;(\Lambda\;\cs).\; \corrv{\rs_\text{v}\;x}{(\envadd{\sigma_2}{\holconst{mk\_rec\_env}\;\css\;\sigma_2})\;x}
    }{
      \corrv{\holconst{Vrecabs}\;\css\;\mathit{name}\;\sigma_1}{\holconst{Vrecabs}\;\css\;\mathit{name}\;\sigma_2}
    }
  \]
  \caption{Right-conflating correspondence (coinductive)}
  \label{code:intermediate:ml:corr}
\end{code}

Both constant definitions and values of variables are recorded in a single environment $\sigma$.
This also applies to the environment contained in a closure.
The correspondence relation thus needs to take a different sets of bindings in closures into account.

Hence, I define a coinductive relation $\approx_\text{v}$ that uses the computed rule set $\rs_\text{v}$ (§\ref{sec:intermediate:value:sublocale}) and compares environments (\cref{code:intermediate:ml:corr}).
I call it \emph{right-conflating}, because in a correspondence $\corrv v u$, any bound environment in $u$ is thought to contain both variables and constants (i.e., the environment is conflated), whereas in $v$, any bound environment contains only variables.
In the correctness property, $v$ and $u$ will originate from the previous and the ML-style semantics, respectively.

It is worth examining the definition of $\approx_\text{v}$:
\begin{semantics}
  \item[Vconstr]
    Constructors are compared structurally.
  \item[Vabs]
    Non-recursive closures must have identical clauses $\mathit{cs}$.
    The bound environment $\sigma_2$ of the right closure is compared on two subsets:
    \begin{itemize}
      \item on the free variables of $\mathit{cs}$, it must correspond to the bound environment $\sigma_1$ of the left closure, and
      \item on the constants of $\mathit{cs}$, it must correspond to the rules $\rs$ of the global environment.
    \end{itemize}
  \item[Vrecabs]
    Recursive closures are treated similarly to non-recursive closures, with one exception:
    In the constants of the clauses set $\mathit{css}$, $\sigma_2$ is first augmented with $\holconst{mk\_rec\_env}\;\css\;\sigma_2$ before relating to \rs.
    This corresponds to the \textsc{RecComb} rule of the semantics.
    It is also the reason why this relation must be coinductive:
    any derivation tree involving recursive closures will be infinite, because the recursive environment keeps being inserted into $\sigma_2$.
\end{semantics}

\noindent
This relation has some perhaps surprising properties.
Most importantly, it is not reflexive.
For example, assuming a constant \holconst{a} is defined, the following proposition does not hold:
%
\[
  \corrv{\holconst{Vabs}\;\cs\;[]}{\holconst{Vabs}\;\cs\;[]} \quad\text{where}\;\cs = [\embed{x} \Rightarrow \embed{\holconst{a}}]
\]
%
Fortunately, $\approx_\text{v}$ fits into the usual scheme of relations:

\begin{corollary}
  $\approx_\text{\normalfont v}$ is a structural value relation (§\ref{sec:terms:types:value:rel}).
\end{corollary}

\noindent
Similarly to $\approx_\text{p}$ (\cref{code:intermediate:elim:corr}), this somewhat technical definition implies a more intuitive property:
\begin{lemma}
  $\corrv{u}{v} \implies \holconst{value\_to\_sterm}\;u = \holconst{value\_to\_sterm}\;v$
\end{lemma}

\noindent
\cref{thm:terms:types:value:rel:collapse} also applies; i.e., for closure-free values, $\approx_\text{v}$ collapses to equality.

\subsection{Correctness}

\begin{theorem}
  \label{thm:intermediate:ml}
  Let $\sigma$ be an environment, $t$ be a closed term and $v$ a value such that $\seval*{\sigma}{t}{v}$.
  If for all constants $x$ occurring in $t$, $\corrv{\rs\;x}{\sigma\;x}$ holds, then there is an $u$ such that $\seval{\rs}{[]}{t}{u}$ and $\corrv{u}{v}$.
\end{theorem}

\noindent As usual, the proof proceeds via induction over the semantics (\cref{code:intermediate:ml}).
It is important to note that the global clause set\index{global clause set} construction (§\ref{sec:intermediate:value:sublocale}) satisfies the preconditions of the theorem:

\begin{lemma}
  If $\mathit{name} \in \holconst{dom}\;\holconst{global\_css}$, then:
  \[
    \corrv{\holconst{Vrecabs}\;\holconst{global\_css}\;\mathit{name}\;[]}{\holconst{Vrecabs}\;\holconst{global\_css}\;\mathit{name}\;[]}
  \]
\end{lemma}

\begin{proof}
  Because $\approx_\text{v}$ is defined coinductively, the proof of this precondition proceeds by coinduction.
  In essence, I have to use the \holconst{Vrecabs} rule of $\approx_\text{v}$.
  But because $\sigma_1 = \sigma_2 = []$, the first premise of the rule is trivial and the second one collapses to a correspondence between $\rs_\text{v}$ and $\holconst{mk\_rec\_env}\;\holconst{global\_css}$.
  This is a direct consequence of \cref{thm:intermediate:value:rs-mk-rec-env}, which shows an equality between those two maps.
\end{proof}