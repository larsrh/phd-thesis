% !TeX spellcheck = en_GB
% !TeX encoding = UTF-8

\section{Naming bound variables}
\label{sec:intermediate:named}

\authorship{
  Some proofs in this section have been contributed by Yu Zhang.

  \medskip
  Portions of this section appear in the AFP entry \citetitle{hupel2019algebra}, authored by \citeauthor{hupel2019algebra}~\cite{hupel2019algebra}.
}

\begin{code}[t]
  \[
    \inferrule*[left=Step]{(\mathit{lhs}, \mathit{rhs}) \in \R \\ \holconst{match}\;\mathit{lhs}\;t = \holconst{Some}\;\sigma}{\rewrite{\R}{t}{\holconst{subst}\;\sigma\;\mathit{rhs}}}
    \quad
    \inferrule*[left=Beta]{ }{\rewrite{\R}{\app{(\Lambda x.\;t)}{t'}}{\holconst{subst}\left[x \mapsto t'\right]\;t}} \quad
  \]
  \[
    \inferrule*[left=Fun]{\rewrite{\R}{t}{t'}}{\rewrite{\R}{\app t u}{\app{t'}{u}}}
    \quad
    \inferrule*[left=Arg]{\rewrite{\R}{u}{u'}}{\rewrite{\R}{\app t u}{\app{t}{u'}}}
  \]
  \caption{Small-step semantics with explicit bound variable names}
  \label{code:intermediate:named}
\end{code}

\noindent
Isabelle uses de~Bruijn indices in the term language for the following two reasons:
First, for substitution, there is no need to rename bound variables.
Second, $\alpha$-equivalent terms are equal.

In implementations of programming languages, these advantages are not required:
Typically, substitutions do not happen inside abstractions, and there is no notion of equality of functions.
CakeML is no exception and therefore it uses named variables.
In this compilation step, de~Bruijn indices are removed.

The \emph{named} semantics is based on the \holtype{nterm} type (\cref{code:terms:intermediate}).
The rules are given in \cref{code:intermediate:named}.
Notably, $\beta$-reduction\index{beta-reduction@$\beta$-reduction} reuses the substitution function.

For the correctness proof, I establish a correspondence between \holtype{term}s and \holtype{nterm}s.
Translation from \holtype{nterm} to \holtype{term} is trivial:
Replace bound variables by the number of abstractions between occurrence and where they were bound in, and keep free variables as they are.
This function is called \holconst{nterm\_to\_term}\holconstindex{nterm\_to\_term}.

The other direction is not unique and requires introduction of \keyword{fresh} names for bound variables.
In the formalization, I have chosen to use a \emph{monad} to produce these names (§\ref{sec:intermediate:named:fresh}).
This function is called \holconst{term\_to\_nterm}\holconstindex{term\_to\_nterm}.

Both functions should behave in such a way that conversion back and forth yields the same term.
For converting from \holtype{term} to \holtype{nterm} and back, this is trivial to ensure (§\ref{sec:intermediate:named:trans}).
However, the other direction is harder, because when converting from \holtype{nterm} to \holtype{term}, the original variable names get lost.

\subsection{The \texttt{fresh} monad}
\label{sec:intermediate:named:fresh}

Generation of fresh names in general can be thought of as picking a string that is not an element of a (finite) set of already existing names.
For Isabelle, the \emph{Nominal} framework \cite{urban2008nominal,urban2013nominal} provides support for reasoning over fresh names, but unfortunately, its definitions are not executable.

Instead, I chose to model generation of fresh names as a monad $\alpha\;\holtype{fresh}$\holtypeindex{fresh} with the following high-level interface:
%
\begin{align*}
  \holtypejudgement{\holconst{Fresh.create}}{&\holtype{string}\;\holtype{fresh}} \\
  \holtypejudgement{\holconst{Fresh.return}}{&\alpha \Rightarrow \alpha \;\holtype{fresh}} \\
  \holtypejudgement{\holconst{Fresh.bind}}{&\alpha\;\holtype{fresh} \Rightarrow (\alpha \Rightarrow \beta \;\holtype{fresh}) \Rightarrow \beta \;\holtype{fresh}} \\
  \holtypejudgement{\holconst{Fresh.run}}{&\alpha\;\holtype{fresh} \Rightarrow \holtype{string}\;\holtype{set} \Rightarrow \alpha}
\end{align*}\holconstindex{Fresh.create}\holconst{Fresh.return}\holconst{Fresh.run}\holconst{Fresh.bind}
%
With this, it becomes possible to write programs using \holkeyword{do}-notation.
Note that such a program only tracks the known names via the argument passed into \holconst{run}; it is not possible to declare further names inside the program.
This is sufficient for the compiler, because the set of all reserved names is known in advance:
I take the union of all known constant names, including constructors.

In the formalization, this is implemented using the state monad, i.e., $\alpha\;\holtype{fresh}$ is a type synonym for a function $\holtype{string} \Rightarrow (\alpha \times \holtype{string})$.
Perhaps counter-intuitively, the state type is just a single \holtype{string} instead of a $\holtype{string}\;\holtype{set}$.
It is used to track the highest used name according to the lexicographic order of strings.

Abstractly, this is modelled as a locale that expects two operations $\holtypejudgement{\holconst{next}}{\alpha \Rightarrow \alpha}$\holconstindex{next} and $\holtypejudgement{\holconst{arb}}{\alpha}$\holconstindex{arb} with the axiom $\holconst{next}\;x > x$.
The high-level \holconst{Fresh.create} operation can then be defined as $\lambda x.\;(\holconst{next}\;x, \holconst{next}\;x)$.

It remains to be explained how \holconst{run} can be implemented, i.e., how the highest used name is computed.
The locale expects $\alpha$ to be a \holclass{linorder}.
Using the $\holtypejudgement{\holconst{Max}}{\alpha \Rightarrow \alpha\;\holtype{set}}$ combinator from Isabelle's library and the \holconst{arb} operation, a derived operation to compute the successor of a set of names can be defined as follows:
%
\begin{lstlisting}
definition Next :: $\alpha$ set $\Rightarrow$ $\alpha$ where
Next //S// = (||if|| //S// = $\emptyset$ ||then|| arb ||else|| next (Max //S//))
\end{lstlisting}\holconstindex{Next}
%
Consequently, \holconst{Fresh.run} is merely an abbreviation for first computing the successor of the input set of names, then running the state function with that successor.

\begin{definition}[Freshness]\index{fresh}
  A name $s'$ is fresh in a set $S$ if all $s \in S$ are smaller than $s'$.
\end{definition}

\begin{lemma}
  $\holconst{Next}\;S$ is fresh in $S$, i.e., $\forall s \in S.\; \holconst{Next}\;S > s$.
\end{lemma}

\begin{corollary}
  $\holconst{Next}\;S \not\in S$.
\end{corollary}

\noindent
This abstract specification of fresh name generation has multiple advantages:
\begin{itemize}
  \item
    There may be multiple implementations for a single type; here, I chose suffixing an underscore to the highest name for computing the successor.
  \item
    All existing combinators and lemmas on the state monad can be reused.
  \item
    Reasoning about freshness of names can be reduced to monotonicity arguments.
\end{itemize}

\noindent
The disadvantage is that it is harder to provide an implementation that uses a fixed scheme for fresh names, e.g.\ $\mathtt{"a"}$ with a suffixed number.
In particular, it requires a modified linear order for names.

\subsection{Translations between \holtype{term} and \holtype{nterm}}
\label{sec:intermediate:named:trans}

\begin{code}[t]
  \begin{lstlisting}[language=Isabelle]
fun term_to_nterm :: string list $\Rightarrow$ term $\Rightarrow$ nterm fresh where
term_to_nterm _ (Const //name//) = Fresh.return (Nconst //name//)
term_to_nterm _ (Free //name//) = Fresh.return (Nvar //name//)
term_to_nterm $\Gamma$ (Bound //n//) = Fresh.return (Nvar ($\Gamma$ ! //n//))
term_to_nterm $\Gamma$ ($\Lambda$ //t//) = ||do|| {
  //n// $\leftarrow$ Fresh.create;
  //e// $\leftarrow$ term_to_nterm (//n// # $\Gamma$) t;
  Fresh.return ($\Lambda$ //n//. //e//)
}
term_to_nterm $\Gamma$ ($t_1$ $\${}$ $t_2$) = ||do|| {
  $e_1$ $\leftarrow$ term_to_nterm $\Gamma$ $t_1$;
  $e_2$ $\leftarrow$ term_to_nterm $\Gamma$ $t_2$;
  Fresh.return ($e_1$ $\${}$ $e_2$)
}

fun nterm_to_term :: string list $\Rightarrow$ nterm $\Rightarrow$ term where
nterm_to_term _ (Nconst //name//) = Const //name//
nterm_to_term $\Gamma$ (Nvar //name//) = ||case|| find_first //name// $\Gamma$ ||of||
    Some //n// $\Rightarrow$ Bound //n//
    None $\Rightarrow$ Free //name//
nterm_to_term $\Gamma$ (//t// $\${}$ //u//) = nterm_to_term $\Gamma$ //t// $\${}$ nterm_to_term $\Gamma$ //u//
nterm_to_term $\Gamma$ ($\Lambda$ //x//. //t//) = $\Lambda$ nterm_to_term (//x// # $\Gamma$) //t//\end{lstlisting}\holconstindex{term\_to\_nterm}\holconstindex{nterm\_to\_term}
  \caption{Translations between \holtype{term}s and \holtype{nterm}s}
  \label{code:intermediate:named:translation}
\end{code}

\noindent
Both translations are recursive functions on the structure of the input terms, with an additional parameter $\Gamma$ that records the context of bound variable names (\cref{code:intermediate:named:translation}).

For terms that contain no abstractions, \holconst{term\_to\_nterm} coincides with \holconst{convert\_term} for all $\Gamma$; the opposite direction coincides only for $\Gamma = []$.

\begin{lemma}
  \label{thm:intermediate:named:translate}
  Let $t$ be a wellformed \holconst{term} (i.e., without dangling de~Bruijn indices).
  Then, the translation of $t$ from \holtype{term} to \holtype{nterm} is reversible:
  \[ \holconst{nterm\_to\_term}\;[]\;(\holconst{run}\;(\holconst{term\_to\_nterm}\;[]\;t)\;S) = t, \]
  where $\holconst{frees}\;t \subseteq S$.
\end{lemma}

\noindent
The basic proof idea is to use induction over $t$ after suitable generalization for an arbitrary context $\Gamma$.

\begin{code}[t]
  \[
    \inferrule*{
    }{\Gamma \vdash \holconst{Nconst}\;x \approx_\alpha \holconst{Nconst}\;x} \quad
    \inferrule*{
      x \not\in \holconst{dom}\;\Gamma \\
      x \not\in \holconst{range}\;\Gamma
    }{\Gamma \vdash \holconst{Nvar}\;x \approx_\alpha \holconst{Nvar}\;x} \quad
    \inferrule*{
      \Gamma\;x = \holconst{Some}\;y
    }{\Gamma \vdash \holconst{Nvar}\;x \approx_\alpha \holconst{Nvar}\;y}
  \]
  \[
    \inferrule*{
      \envadd\Gamma{[ x \mapsto y ]} \vdash t \approx_\alpha u
    }{\Gamma \vdash \Lambda x.\; t \approx_\alpha \Lambda y.\; u} \quad
    \inferrule*{
      \Gamma \vdash t_1 \approx_\alpha u_1 \\
      \Gamma \vdash t_2 \approx_\alpha u_2
    }{\Gamma \vdash \app{t_1}{t_2} \approx_\alpha \app{u_1}{u_2}}
  \]
  \caption{$\alpha$-equivalence between terms}
  \label{code:intermediate:named:alpha}
\end{code}

As already indicated earlier, the correctness property of the opposite direction requires $\alpha$-equivalence.
Its definition is given in \cref{code:intermediate:named:alpha}.
A slight deviation from literature is that I use an explicit context of renamings instead of substituting directly.
This is merely for convenience.
Two terms can be said to be $\alpha$-equivalent if there is a context of renamings that transforms one into the other.

\begin{corollary}[Reflexivity]\label{thm:intermediate:named:refl}
  For all terms $t$, $[] \vdash t \approx_\alpha t$ holds.
\end{corollary}

\noindent
The correctness property can now be phrased accordingly.
Because the term that is translated may not be closed, it is necessary for this direction to take contexts into account:

\begin{lemma}
  Let $t$ be a \holtype{nterm} and $\Gamma$, $\Gamma'$ be contexts with the same length and $\holconst{frees}\;t \subseteq \Gamma$.
  Then, the translation of $t$ from \holtype{nterm} to \holtype{term} (with context $\Gamma'$) and back (with context $\Gamma$) is $\alpha$-equivalent to $t$.
  The corresponding renaming is the pairing of $\Gamma$ and $\Gamma'$.
  Formally:
  \[ \holconst{map\_of}\;(\holconst{zip}\;\Gamma\;\Gamma') \vdash \holconst{Fresh.run}\;(\holconst{term\_to\_nterm}\;\Gamma'\;(\holconst{nterm\_to\_term}\;\Gamma\;t))\;S \approx_\alpha t \]
\end{lemma}

\noindent
Note that this lemma also constructs the renaming.
For closed terms, the following simplified corollary can be obtained:

\begin{corollary}
  Let $t$ be a closed \holtype{nterm}.
  Then, the translation of $t$ from \holtype{nterm} to \holtype{term} and back is $\alpha$-equivalent to $t$:
  \[ [] \vdash \holconst{Fresh.run}\;(\holconst{term\_to\_nterm}\;[]\;(\holconst{nterm\_to\_term}\;[]\;t))\;S \approx_\alpha t \]
\end{corollary}

\noindent
Furthermore, it is possible to prove that two translation runs from \holconst{term} to \holconst{nterm} with different contexts yield $\alpha$-equivalent terms.

\begin{lemma}
  Let $\Gamma_1$ and $\Gamma_2$ be two contexts with the same length such that both $\Gamma_1$ and $\Gamma_2$ have no duplicate elements.
  Let $t$ be a closed \holtype{term}.
  Let $s_1$ and $s_2$ be names that are fresh in $\Gamma_1$ and $\Gamma_2$, respectively.
  Then:
  \[ \holconst{map\_of}\;(\holconst{zip}\;\Gamma_1\;\Gamma_2) \vdash \holconst{fst}\;(\holconst{term\_to\_nterm}\;\Gamma_1\;t\;s_1) \approx_\alpha \holconst{fst}\;(\holconst{term\_to\_nterm}\;\Gamma_2\;t\;s_2) \]
\end{lemma}
%
\remark*{
  In the above statement, \holconst{term\_to\_nterm} is used as a function with three arguments, because \holtype{fresh} is implemented as a state monad, which is in turn a function of one argument.
  In this case, the third argument is a simple \holtype{string}.
  \holconst{Fresh.run} is not used, because it expects a $\holtype{string}\;\holtype{set}$.

  An instantiated version of this lemma with $\Gamma_1 = \Gamma_2 = []$ would make little sense: it yields a special case of \cref{thm:intermediate:named:refl}.
}

\noindent
As usual, the proof proceeds by generalization and subsequent induction over the term.

In the remainder of this section, I will explain the relationship between terms and their translations with respect to matching and substitution.

\begin{definition}\label{def:intermediate:named:rel}
  A \holtype{term} $t$ and an \holtype{nterm} $u$ are related wrt $\Gamma$ if $t = \holconst{nterm\_to\_term}\;\Gamma\;u$.
\end{definition}

\begin{corollary}\label{thm:intermediate:named:rel}
  \cref{def:intermediate:named:rel} is a strong structural term relation (§\ref{sec:terms:algebra:thy:preds}) for all $\Gamma$.
\end{corollary}

\noindent
From this corollary, it follows that matching behaves identically on terms and their translations.
However, there are no generic lemmas over substitution (§\ref{sec:terms:algebra:thy:preds}).

\begin{lemma}\label{thm:intermediate:named:subst}
  Let $\sigma_1$ and $\sigma_2$ be closed and related environments wrt $\Gamma$.
  If $\Gamma$ and the domain of $\sigma_1$ are disjoint, then \holconst{nterm\_to\_term} and \holconst{subst} commute.
  Formally:
  \[ \holconst{subst}\;\sigma_1\;(\holconst{nterm\_to\_term}\;\Gamma\;t) = \holconst{nterm\_to\_term}\;\Gamma\;(\holconst{subst}\;\sigma_2\;t) \]
\end{lemma}

\noindent
The proof proceeds by induction on $t$ and crucially depends on the $\sigma_i$ being closed.

The next lemma establishes a connection between $\beta$-reduction (§\ref{sec:background:rewriting}) and substitution:

\begin{lemma}\label{thm:intermediate:named:beta}
  If $t'$ is closed, then:
  \[
    \holconst{nterm\_to\_term}\;\Gamma\;(\holconst{subst}\;\left[x \mapsto t'\right]\;t) =
    (\holconst{nterm\_to\_term}\;(x \cons \Gamma)\;t) [\holconst{nterm\_to\_term}\;\Gamma\;t']
  \]
\end{lemma}

\noindent
Informally speaking, this means that one can either substitute $t'$ for $x$ in $t$ and then translate the resulting term back, or translate both $t$ and $t'$ back and then perform $\beta$-reduction.

\subsection{Compilation, correctness, \& completeness}
\label{sec:intermediate:named:compilation}

Having established a low-level translation function from \holtype{term} to \holtype{nterm} that invents bound variable names based on names from a context, it remains to explain how full rule sets are compiled.
The idea is rather straightforward:
translate the right-hand side of all rules while leaving the left-hand sides unchanged.
The left-hand side is still represented as \holtype{term}, for the reason outlined in §\ref{sec:terms:algebra:matching}.

Formally, compilation is defined as follows:
\[
  \holconst{compile}\;\R = \{ (p, \holconst{Fresh.run}\;(\holconst{term\_to\_nterm}\;[]\;t)\;(\holconst{all\_consts} \cup \holconst{frees}\;p)) \mid (p, t) \in \R \}
\]
%
The fresh monad receives two sets of known names.
Naturally, names bound by the patterns must be avoided ($\holconst{frees}\;p$).
To avoid collisions of names further down in the compilation pipeline, additionally all the known constant names are supplied (\holconst{all\_consts}).
Recall that that set comprises two subsets (\cref{code:deep:constructors}):
\begin{itemize}
  \item the names of all constructors from the \holconst{constructors} locale
  \item the names of all equation heads ($\holconst{heads\_of}\;\R$)
\end{itemize}
%
The side conditions of the locale (§\ref{sec:intermediate:wellformed}) ensure that $\holconst{heads\_of}\;\R$ is equal to the set \holconst{heads} defined during deep embedding (\cref{code:deep:constructors}).

\begin{corollary}[Invariance of heads]
  The heads of $\R$ remain unchanged after compilation.
\end{corollary}

\begin{theorem}[Correctness of compilation]\label{thm:intermediate:named}
  Assuming a step can be taken with the compiled rule set, this step can be reproduced with the original rule set.
  \[
    \inferrule{
      \rewrite{\holconst{compile}\;\R}{t}{t'} \\
      \holconst{closed}\;t
    }{
      \rewrite\R{\holconst{nterm\_to\_term}\;t}{\holconst{nterm\_to\_term}\;t'}
    }
  \]
\end{theorem}

\begin{proof}
  By rule induction over the semantics (\cref{code:intermediate:named}).
  The interesting cases are \textsc{Step} and \textsc{Beta}.

  \begin{induction}
    \item[Step]
      This follows from \cref{thm:intermediate:named:rel,thm:intermediate:named:subst}.
    \item[Beta]
      This follows from \cref{thm:intermediate:named:beta}. \qedhere
  \end{induction}
\end{proof}

\begin{theorem}[Completeness of compilation]\label{thm:intermediate:named-compl}
  Assuming a step can be taken in the original rule set, this step can be reproduced in the compiled rule set to obtain an $\alpha$-equivalent term.
  Formally: if $\rewrite\R{t}{t'}$ where $t$ is closed and wellformed, then there is a term $u'$ such that:
  \[
    \rewrite{\holconst{compile}\;\R}{\holconst{term\_to\_nterm}\;[]\;t\;s}{u'} \wedge [] \vdash u' \approx_\alpha \holconst{term\_to\_nterm}\;[]\;t'\;s'
  \]
\end{theorem}
