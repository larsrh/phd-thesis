% !TeX spellcheck = en_GB
% !TeX encoding = UTF-8

\section{Related work}

There is existing work in the Coq \cite{anand2017certicoq,forster2016extraction} and HOL \cite{myreen2014translation} communities for proof producing or verified extraction of functions defined in the logic.
Anand \etal\ \cite{anand2017certicoq} present work in progress on a verified compiler from Gallina (Coq's specification language) via untyped intermediate languages to CompCert C light.
They plan to connect their extraction routine to the CompCert compiler \cite{leroy2009compiler}.

Compilation of pattern matching is well understood in literature \cite{augustsson1985pattern,peytonjones1987implementation,slind1999terminating}.
In this work, I contribute a transformation of sets of equations with pattern matching on the left-hand side into a single equation with nested pattern matching on the right-hand side. This is implemented and verified inside Isabelle.

Besides CakeML, there are many projects for verified compilers for functional programming languages of various degrees of sophistication and realism~\cite{BentonH09,Chlipala10,Flatau}.
Particularly modular is the work by Neis \etal\ \cite{neis2015pilsner} on a verified compiler for the ML-like imperative source language Pilsner.
The main distinguishing feature of this work is that I start from a set of higher-order recursion equations with pattern matching on the left-hand side rather than a lambda calculus with pattern matching on the right-hand side.
On the other hand I stand on the shoulders of CakeML which allows me to bypass all complications of machine code generation.
Note that much of the formalization is not specific to CakeML and that it would be possible to retarget it to, for example, Pilsner abstract syntax with moderate effort.