% !TeX spellcheck = en_GB
% !TeX encoding = UTF-8

\section{CakeML}
\label{sec:intermediate:cakeml}

CakeML's semantics has been formalised in Lem \cite{mulligan2014lem}.
For the correctness proof of the CakeML compiler, its authors have extracted this into HOL theories.
In this work, I directly target CakeML abstract syntax trees (thereby bypassing the parser) and use its big-step semantics, which I have extracted into Isabelle~\cite{hupel2018cakeml}.

Consequently, there is not a single correctness result in Isabelle, but rather two parts:
A frontend from Isabelle to CakeML, and a backend from CakeML to machine code, the latter of which is provided by the ongoing work on CakeML~\cite{tan2016cakeml}.

In order to execute the Isabelle-generated CakeML syntax trees, there are three choices:
\begin{enumerate}
  \item
    turning CakeML's big step semantics into an executable function within Isabelle using the predicate compiler \cite{berghofer2009inductive},
  \item
    executing the functional big-step semantics directly within Isabelle \cite{owens2016functional}, and
  \item
    printing the tree as an s-expression and compiling it with the official CakeML compiler.%
    \thyrefafp*{CakeML}{CakeML_Compiler}
\end{enumerate}

\subsection{CakeML's semantic functions}

In §\ref{sec:background:cakeml}, I have outlined the work that was required to integrate the Lem export of CakeML with Isabelle.
Notwithstanding that, the CakeML theories describe an entirely separate universe from the definitions in §\ref{sec:terms}.
Notably, it is not possible to instantiate the \holclass{term} class for CakeML expressions.
In this section, I will discuss the differences between my and CakeML's implementation of term operations.

Almost all of CakeML's semantic functions take (a fragment of) an environment into account.
An environment consists of two \emph{namespaces}: the \emph{value} and the \emph{constructor} namespace.
The value namespace records existing bindings.
It is comparable to the $\Gamma$ and $\sigma$ used in the various intermediate semantics.
The constructor namespace keeps track of defined datatypes and their constructors, including their arities.

\subsubsection{CupCakeML: A purely-functional subset of CakeML}
\label{sec:intermediate:cakeml:cupcake}

The core of CakeML's big-step semantics is organized in three mutually recursive inductive predicates (evaluating an expression, a list of expressions, and a match expression) with a total of over thirty rules.
It is that large because it has to deal with exceptions, modules, step counters, and other features that their compiler supports, but I do not need in the formalization.
For that reason, I have defined the subset \emph{CupCakeML} which only allows the syntactic forms known from the \holtype{sterm} type.
It also enforces that all occurring values have an empty module environment and a consistent datatype environment.

I define a smaller big-step semantics comprising a single inductive predicate with twelve cases.
Then, I prove equivalence to the original semantics, that is, both correctness and completeness, under the assumption that the expression and initial environments are in the supported fragment.
The supported fragment has also been proved to be closed under evaluation.

The namespaces are allowed to vary in closures, since CakeML supports introducing new type definitions inside a program.
Tracking this would greatly complicate the proofs.
Hence, I assume a fixed constructor namespace and enforce that all values use exactly that one.
This information is provided by the \holconst{C\_info} parameter (\cref{code:deep:constructors}) that is generated during embedding.

\subsubsection{Patterns and matching}

\begin{code}
  \begin{lstlisting}[language=Isabelle]
fun pat_to_cake :: pat $\Rightarrow$ Cake.pat where
pat_to_cake (Patvar //s//) = Cake.Pvar //s//
pat_to_cake (Patconstr //s// //args//) =
  Cake.Pcon (Some (Cake.Short //s//)) (map pat_to_cake args)\end{lstlisting}\holconstindex{pat\_to\_cake}
  \caption{Translation between patterns}
  \label{code:intermediate:cakeml:pat}
\end{code}

CakeML defines three distinct datatypes for patterns, expressions and values.
Because they model the syntax of a real-world programming language, they are much more complex than the types required for term rewriting.

The most notable difference in types is that patterns and constructor values in CakeML are $n$-ary.
This corresponds closely to the \holtype{value} (§\ref{sec:terms:types:value}) and \holtype{pat} (§\ref{sec:terms:types:pat}) types that I introduced for the evaluation semantics (§\ref{sec:intermediate:value}).
Consequently, a conversion function between the pattern types is easily defined (\cref{code:intermediate:cakeml:pat}).
Additional CakeML complexities, for example pattern wildcards ($\holkeyword{case}\;t\;\holkeyword{of}\;\_ \Rightarrow u$) or qualified constructor names (\holconst{List.Cons}) are excluded from the CupCakeML fragment.

CakeML's matching function has three possible results: \holconst{Match}, \holconst{No\_match}, and \holconst{Match\_type\_\allowbreak{}error}.
The latter result may occur when the pattern and the object disagree about their arity.
For example, consider the ill-typed program $\holkeyword{case}\;\holconst{T}\;x\;\holkeyword{of}\;\holconst{T} \Rightarrow y$.
In my implementation, matching $\holconst{T}\;x$ to \holconst{T} would result in \holconst{None}.
However, in CakeML it returns \holconst{Match\_type\_error}.
A similar check happens when constructing constructor values, too.
In short, data constructors must always occur with all arguments supplied on right-hand and left-hand sides.

During embedding (§\ref{sec:deep}), all type information is erased.
Fully applied constructors in terms can be easily guaranteed by the introduction of constructor functions (§\ref{sec:preproc:constructor}).
For patterns however, this must be ensured throughout the compilation pipeline; it is (like other syntactic constraints) another side condition imposed on the rule set (§\ref{sec:intermediate:wellformed}).

For the purposes of the correctness proof, a CakeML type error is treated equivalently to non-termination.
This means that should the evaluation of a program end in a type error, there are no guarantees.
However, I avoid this problem by running the CakeML type checker on the generated programs (§\ref{sec:conclusion:eval}).

An additional complication comes from the shallow nature of the typing check.
Not all ill-typed constructors yield a type error.
For example, matching the object $\holconst{T}\;\holconst{U2}\;(\holconst{V}\;x)$ against the pattern $\holconst{T}\;\holconst{U1}\;\holconst{V}$ would result in \holconst{No\_match}.
\holconst{U2} and \holconst{U1} do not match.
At that point, the matching function does not proceed further and the arity mismatch in the second argument remains hidden.

Similarly to my implementation of matching, CakeML does not check for linearity of patterns.
However, the linearity check happens in the big-step semantics, so it needs to be dealt with there.

\subsubsection{Closures}

Closures are syntactically similar between the two different semantics.
In particular, the \holconst{mk\_rec\_env} function I introduced for the ML-style evaluation semantics (§\ref{sec:intermediate:ml}) has been modelled after its counterpart in the CakeML semantics.

It remains to deal with application of closures to values.
CakeML models function application as a binary operator, similarly to other operators such as addition on machine words.
The CupCakeML fragment enforces that application is the only operator that may occur in generated code, since my formalization never emits operations on machine types.
The complication here is that in my semantics, application always requires matching.
In CakeML, closures always take exactly one argument which is not matched, but directly bound as a variable.
The body then may or may not perform a matching step.
I will explain this later when introducing an appropriate correspondence relation (\cref{code:intermediate:cakeml:v-corr}).

\subsection{Translation from \holtype{sterm} to \holtype{exp}}

\begin{code}
  \begin{lstlisting}[language=Isabelle]
fun
  sterm_to_cake :: string set $\Rightarrow$ sterm $\Rightarrow$ Cake.exp and
  clauses_to_cake :: string set $\Rightarrow$ (term $\times$ sterm) list $\Rightarrow$ (Cake.pat $\times$ Cake.exp) list
where
sterm_to_cake _ (Svar //s//) = Cake.Var (Cake.Short //s//)
sterm_to_cake _ (Sconst //s//) = Cake.Var (Cake.Short //s//)
sterm_to_cake //S// ($t_1$ $\$$ $t_2$) =
  Cake.App Cake.Opapp [sterm_to_cake //S// $t_1$, sterm_to_cake //S// $t_2$]
sterm_to_cake //S// (Sabs //cs//) =
  ||let|| //n// = Fresh.run Fresh.create (//S// $\cup$ constructors) ||in||
    Cake.Fun //n// (Cake.Mat (Cake.Var (Cake.Short //n//)) (clauses_to_cake //S// //cs//))
clauses_to_cake //S// //cs// =
  [(pat_to_cake (mk_pat //pat//), sterm_to_cake (frees //pat// $\cup$ //S//) //t//)) | (//pat//, //t//) $\leftarrow$ //cs//]\end{lstlisting}\holconstindex{sterm\_to\_cake}\holconstindex{clauses\_to\_cake}

  \caption{Translation from \holtype{sterm}s to \holtype{exp}s}
  \label{code:intermediate:cakeml:translation}
\end{code}

After the series of translations described in the earlier sections, the terms in my formalization are syntactically close to CakeML's terms (\holtype{Cake.exp}).
The only remaining differences are outlined below:

\begin{itemize}
  \item
    CakeML does not combine abstraction and pattern matching.
    For that reason, I need to translate $\Lambda\;[p_1 \Rightarrow t_1, \ldots]$ into $\Lambda x.\;\holkeyword{case}\;x\;\holkeyword{of}\;p_1 \Rightarrow t_1 \mid \ldots$, where $x$ is a fresh variable name.
    I reuse the \holtype{fresh} monad (§\ref{sec:intermediate:named:fresh}) to obtain a bound variable name.
  \item
    CakeML has two distinct syntactic categories for identifiers\index{identifier} (that can represent variables or constants) and data constructors.
    My term types however have two distinct syntactic categories for constants (that can represent function definitions or data constructors) and variables.
    The necessary prerequisites to deal with this are already present in the ML-style evaluation semantics (§\ref{sec:intermediate:ml}) which conflates constants and variables, but has a dedicated \textsc{Constr} rule for data constructors.
\end{itemize}

\noindent
The corresponding translation functions are given in \cref{code:intermediate:cakeml:translation} (the type prefix \holtype{Cake} indicates a CakeML type).

In the \holconst{Sabs} case, is not necessary to thread through already created variable names, only existing names.
The reason is simple: a generated variable is bound and then immediately used in the body.
Shadowing the name somewhere in the body does not cause any problems.
This is in contrast to the \holconst{term\_to\_nterm} function (\cref{code:intermediate:named:translation}), which needs to track all bound variable names.

The $S$ parameter then needs to be initialized with all existing constants; similarly to §\ref{sec:intermediate:named:compilation}, this can be achieved with \holconst{all\_consts}.

\subsection{Correspondence relations}

\begin{code}[t]
  \[
    \inferrule*[left=Var]{
    }{
      \cakerelexp{\holconst{Svar}\;n}{\holconst{Cake.Var}\;n}
    }
    \quad
    \inferrule*[left=Const]{
      n \notin \holconst{C}
    }{
      \cakerelexp{\holconst{Sconst}\;n}{\holconst{Cake.Var}\;n}
    }
  \]
  \[
    \inferrule*[left=Constr]{
      n \in \holconst{C} \\
      \holconst{rel}\;(\approx_\text{\cakesym})\;\mathit{ts}\;\mathit{us}
    }{
      \cakerelexp{\listcomb{\mathit{name}}{\mathit{ts}}}{\holconst{Cake.Con}\;(\holconst{Some}\;(\holconst{Cake.Short}\;\mathit{name})\;\mathit{us})}
    }
  \]
  \[
    \inferrule*[left=App]{
      \cakerelexp{t_1}{u_1} \\
      \cakerelexp{t_2}{u_2}
    }{
      \cakerelexp{\app{t_1}{t_2}}{\holconst{Cake.App}\;\holconst{Cake.Opapp}\;[u_1, u_2]}
    }
  \]
  \[
    \inferrule*[left=Fun]{
      n \notin \holconst{ids}\;(\Lambda\;\cs) \\
      n \notin \holconst{all\_consts} \\
      \holconst{rel}_\holtype{list}\;(\holconst{rel}_\holtype{prod}\;(\lambda\;t\;p.\; p = \holconst{term\_to\_pat}\;(\holconst{pat\_to\_cake}\;t))\;(\approx_\text{\cakesym}))\;\cs\;\mathit{ml\_cs}
    }{
      \cakerelexp{\Lambda\;\cs}{\holconst{Cake.Fun}\;n\;(\holconst{Cake.Mat}\;(\holconst{Cake.Var}\;n))\;\mathit{ml\_cs}}
    }
  \]
  \[
    \inferrule*[left=Mat]{
      \cakerelexp{t}{u} \\
      \holconst{rel}_\holtype{list}\;(\holconst{rel}_\holtype{prod}\;(\lambda\;t\;p.\; p = \holconst{term\_to\_pat}\;(\holconst{pat\_to\_cake}\;t))\;(\approx_\text{\cakesym}))\;\cs\;\mathit{ml\_cs}
    }{
      \cakerelexp{\app{\Lambda\;\cs}{t}}{\holconst{Cake.Mat}\;u\;\mathit{ml\_cs}}
    }
  \]

  \caption{Expression correspondence}
  \label{code:intermediate:cakeml:exp-corr}
\end{code}

I define two different correspondence relations: one for expressions and one for values.
Both share the $\approx_\text{\cakesym}$ operator (where \cakesym*{} is the official CakeML icon), because they can be disambiguated by types.
There is no separate relation for patterns, because their translation is simple.

First, I will examine the expression correspondence (\cref{code:intermediate:cakeml:exp-corr}).
%
\begin{semantics}
  \item[Var]
    Variables are directly related by identical name.
  \item[Const]
    Recall that constructors are treated specially in CakeML.
    In order to not confuse functions or variables with data constructors themselves, the relation demands that the constant name is not a constructor.
  \item[Constr]
    Constructors are directly related by identical name, and recursively related arguments.
  \item[App]
    CakeML does not just support general function application but also unary and binary operators.
    In fact, function application is represented by the binary operator \holconst{Opapp}.
    The \holconst{sterm\_to\_exp} translation function never generates other operators.
    Consequently, the correspondence is restricted to \holconst{Opapp}.
  \item[Fun/Mat]
    Observe the symmetry between these two cases:
    In my term language, matching and abstraction are combined, which is not the case in CakeML.
    This means relating a case abstraction to a CakeML function containing a match, and a case abstraction applied to a value to just a CakeML match.
    The additional requirement is that the bound variable name in a CakeML function must not occur in the clauses nor be a known constant name (\cref{code:deep:constructors}).
\end{semantics}

\begin{code}[t]
  \[
    \inferrule*[left=Constr]{
      \holconst{rel}\;(\approx_\text{\cakesym})\;\mathit{vs}\;\mathit{us}
    }{
      \cakerelval{\holconst{Vconstr}\;\mathit{name}\;\mathit{vs}}{\holconst{Cake.Conv}\;(\holconst{Some}\;(\mathit{name}, \mathit{type})\;\mathit{us})}
    }
  \]
  \[
    \inferrule*[left=Abs]{
      n \notin \holconst{ids}\;(\Lambda\;\cs) \\
      n \notin \holconst{all\_consts} \\
      \holconst{rel}_\holtype{list}\;(\holconst{rel}_\holtype{prod}\;(\lambda\;t\;p.\; p = \holconst{term\_to\_pat}\;(\holconst{pat\_to\_cake}\;t))\;(\approx_\text{\cakesym}))\;\cs\;\mathit{ml\_cs} \\
      \forall x \in \holconst{ids}\;(\Lambda\;\cs).\;
        \cakerelval{\Gamma\;x}{\holconst{map\_of}\;\mathit{env}\;x}
    }{
      \cakerelval{
        \holconst{Vabs}\;\cs\;\Gamma
      }{
        \holconst{Cake.Closure}\;\mathit{env}\;n\;(\holconst{Cake.Mat}\;(\holconst{Cake.Var}\;n)\;\mathit{ml\_cs})
      }
    }
  \]
  \caption{Value correspondence}
  \label{code:intermediate:cakeml:v-corr}
\end{code}

\noindent
The value correspondence is structurally simpler, but unfortunately, syntactically noisier (\cref{code:intermediate:cakeml:v-corr}).

\begin{semantics}
  \item[Constr]
    Constructor values are compared recursively.
  \item[Abs]
    Comparing closures has multiple requirements:
    \begin{itemize}
      \item the CakeML closure must take a parameter with the bound variable name $n$ that is immediately matched,
      \item $n$ must be a fresh variable,
      \item the clauses must correspond according to the expression correspondence, and
      \item the environments must correspond recursively, but only on the identifiers that occur in the clauses.
    \end{itemize}
  \item[RecAbs]
    This case is identical to non-recursive closures, but lifted to lists of clauses.
    It is omitted \cref{code:intermediate:cakeml:v-corr} because of its syntactic size.
\end{semantics}

\subsection{Correctness}

\begin{lemma}[Correctness of \holconst{sterm\_to\_cake}]
  \[
    \cakerelexp{t}{\holconst{sterm\_to\_cake}\;(\holconst{ids}\;t \cup \holconst{all\_consts})\;t}
  \]
\end{lemma}

\begin{lemma}[Composition of value and extensional correspondence]\label{thm:intermediate:cake:ext}
  \[ \cakerelval{v}{u} \wedge \corre{v'}{v} \implies \cakerelval{v'}{u} \]
\end{lemma}

\noindent
I use the same trick as in §\ref{sec:intermediate:value:sublocale} to obtain a suitable environment for CakeML evaluation based on the rule set $\rs$.
Assuming this environment, the correctness theorem can be stated as follows:

\begin{theorem}[Correctness]
  If a CakeML expression $e$ terminates with a value $u$ in the CakeML semantics and $\cakerelexp{t}{e}$ there is a value $v$ such that $\cakerelval{v}{u}$ and $t$ evaluates to $v$.
\end{theorem}

\begin{proof}
  The proof proceeds by induction on the CakeML evaluation.
  I will discuss some of the ideas here.

  The most interesting case is application.
  By assumption, $t = \app{t_1}{t_2}$ (similarly for $e$).
  By induction hypothesis, $t_1$ evaluates to a (possibly recursive) closure.
  I need to show that $t$ evaluates to a value $\holtypejudgement{v}{\holtype{value}}$ corresponding to $u$.

  Assume that $e_1$ evaluates to a non-recursive closure.
  The recursive case is similar.

  Consequently, according to \cref{code:intermediate:cakeml:exp-corr}, the closure must take a variable with a fresh name and immediately match it against a list of clauses.
  This additional variable gets added to the environment; hence, \cref{thm:intermediate:ml:ext,thm:intermediate:cake:ext} must be used to establish correspondence.
\end{proof}