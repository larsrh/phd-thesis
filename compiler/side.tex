% !TeX spellcheck = en_GB
% !TeX encoding = UTF-8

\section{Side conditions}
\label{sec:intermediate:wellformed}

All of the semantics presented in this chapter require some side conditions on the rule set.
These conditions are purely syntactic.
For example, these are the conditions for the correctness of the first compiler phase:

\begin{itemize}
  \item Patterns must be linear, and constructors in patterns must be fully applied.
  \item Definitions must have at least one parameter on the left-hand side (§\ref{sec:intermediate:value}).
  \item The right-hand side of an equation refers only to free variables occurring in patterns on the left-hand side and contains no dangling de~Bruijn indices.
  \item There are no two defining equations $\mathit{lhs} = \mathit{rhs}_1$ and $\mathit{lhs} = \mathit{rhs}_2$ such that $\mathit{rhs}_1 \neq \mathit{rhs}_2$.
  \item For each pair of equations that define the same constant, their arity must be equal and their patterns must be compatible (§\ref{sec:intermediate:elim}).
  \item There is at least one equation.
  \item Variable names occurring in patterns must not overlap with constant names (§\ref{sec:intermediate:ml}).
  \item Any occurring constants must either be defined by an equation or be a constructor.
\end{itemize}

\noindent
The conditions for the subsequent phases are sufficiently similar that I do not list them again.

In the formalization, I use locales to fix the rules and assumptions over them.
Each phase has its own locale, together with a sublocale proof that after compilation, the conditions are preserved in the new semantics.
Some of the correspondence relations are defined within the locale, so that they can access the fixed rules (and potentially additional parameters).
For technical reasons, the semantics are usually defined outside of the locales.
In the subsequent sections, the modelling as locales and the sublocale proofs are largely ignored, since they offer no new insights into the formalization.