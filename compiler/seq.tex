% !TeX spellcheck = en_GB
% !TeX encoding = UTF-8

\section{Sequentialization}
\label{sec:intermediate:seq}

\authorship{
  Some proofs in this section have been contributed by Yu Zhang.
}

\begin{code}[t]
  \[
    \inferrule*[left=Step]{(\mathit{name}, \mathit{rhs}) \in \rs}{\rewrite{\rs}{\holconst{Sconst}\;\mathit{name}}{\mathit{rhs}}}
    \quad
    \inferrule*[left=Beta]{\holconst{find\_match}\;\cs\;t = \holconst{Some}\;(\sigma, \mathit{rhs})}{\rewrite{\rs}{\app{(\Lambda\; \cs)}{t}}{\holconst{subst}\;\sigma\;\mathit{rhs}}}
  \]
  \[
    \inferrule*[left=Fun]{\rewrite{\rs}{t}{t'}}{\rewrite{\rs}{\app t u}{\app{t'}{u}}}
    \quad
    \inferrule*[left=Arg]{\rewrite{\rs}{u}{u'}}{\rewrite{\rs}{\app t u}{\app{t}{u'}}}
  \]
  \caption{Small-step semantics with ordered clauses}
  \label{code:intermediate:seq}
\end{code}

\noindent
The semantics of \holtype{pterm} and \holtype{sterm} (\cref{code:intermediate:seq}) differ only in rule \textsc{Beta}.
Instead of any matching clause, the first matching clause in a case abstraction is picked.
For technical reasons, the \textsc{Step} rule is phrased using an additional inductive relation in the formalization.
Because the rule set is now a rule list, I use the naming convention $\rs$ instead of \R.

For the correctness proof, the order of clauses does not matter:
I only need to prove that a step taken in the sequential semantics can be reproduced in the unordered semantics.
As long as no rules are dropped, this is trivially true.
For that reason, the compiler can choose an arbitrary ordering.

\remark*{
  This semantics only sequentializes pattern matching:
  Rewriting may still be non-deterministic because if there are multiple redexes in a term, either can be picked.
}

\subsection{Translations between \holtype{pterm} and \holtype{sterm}}

\begin{code}[t]
  \begin{lstlisting}[language=Isabelle]
fun pterm_to_sterm :: pterm $\Rightarrow$ sterm where
pterm_to_sterm (Pconst //name//) = Sconst //name//
pterm_to_sterm (Pvar //name//) = Svar //name//
pterm_to_sterm (//t// $\${}$ //u//) = pterm_to_sterm //t// $\${}$ pterm_to_sterm //u//
pterm_to_sterm (Pabs //cs//) = Sabs (ordered_map {(//p//, pterm_to_sterm //t//) | (//p//, //t//) $\in$ //cs//})

fun sterm_to_pterm :: sterm $\Rightarrow$ pterm where
sterm_to_pterm (Sconst //name//) = Pconst //name//
sterm_to_pterm (Svar //name//) = Pvar //name//
sterm_to_pterm (t $\${}$ u) = sterm_to_pterm //t// $\${}$ sterm_to_pterm //u//
sterm_to_pterm (Sabs //cs//) = Pabs (set (map ($\holconst{map}_\holtype{prod}$ id untranslate_sterm) //cs//))\end{lstlisting}\holconstindex{pterm\_to\_sterm}\holconstindex{sterm\_to\_pterm}
  \caption{Translations between \holtype{pterm}s and \holtype{sterm}s}
  \label{code:intermediate:seq:convert}
\end{code}

Similarly to §\ref{sec:intermediate:named:trans}, both directions are recursive functions on the structure of the input terms (\cref{code:intermediate:seq:convert}).
However, no name context is required.
For terms that contain no abstractions, both directions also coincide with \holconst{convert\_term}.

I will first explain the technically more interesting conversion from \holtype{pterm} to \holtype{sterm}, as this requires imposing an ordering on the unordered set of clauses.
For simplicity, I have chosen to use the lexicographic order on the patterns, i.e., \holtype{term}s.
Then, the problem can be generalized to finding a function $((\holtypejudgement\alpha{\holclass{linorder}}) \times \beta)\;\holtype{set} \Rightarrow (\alpha \times \beta)\;\holtype{list}$.
%
\begin{lstlisting}[language=Isabelle]
definition ordered_map :: ($\alpha$::linorder $\times$ $\beta$) set $\Rightarrow$ ($\alpha$ $\times$ $\beta$) list where
ordered_map //S// = [
  (//k//, the_elem {(//k'//, //v//) | (//k'//, //v//) $\in$ //S// $\wedge$ //k// = //k'//}) |
  //k// $\leftarrow$ sorted_list_of_set {//k// | (//k//, //v//) $\in$ //S//})
]
\end{lstlisting}\holconstindex{ordered\_map}
%
\noindent
The functions $\holtypejudgement{\holconst{the\_elem}}{\alpha\;\holtype{set} \Rightarrow \alpha}$ and $\holtypejudgement{\holconst{sorted\_list\_of\_set}}{\holtypejudgement\alpha{\holclass{linorder}}}\;\holtype{set} \Rightarrow \alpha\;\holtype{list}$ are defined in the Isabelle library and return the element of a singleton set and the sorted list of elements of a set.
Using these functions, \holconst{ordered\_map} first produces an ordered list of keys, which are then associated with the values from the input set.

Naturally, this requires that the input set is a set of pairs where the first elements are unique.
This is guaranteed by a side condition (§\ref{sec:intermediate:wellformed}).
Otherwise, the result is unspecified.
%
\begin{lemma}
  For suitable $S$, $\holconst{set}\;(\holconst{ordered\_map}\;S) = S$.
\end{lemma}
%
\noindent
The opposite direction is much simpler and has already been explained in §\ref{sec:terms:types:sterm}.
%
\begin{corollary}
  For wellformed $t$, $\holconst{sterm\_to\_pterm}\;(\holconst{pterm\_to\_sterm}\;t) = t$.
\end{corollary}
%
\noindent
The reverse property does not hold directly, because an \holtype{sterm} may contain non-ordered clause lists.
This property would require some form of $\alpha$-equivalence that disregards ordering.

Similarly to §\ref{sec:intermediate:named:trans}, I have used the translation functions to establish term correspondence.

\begin{definition}\label{def:intermediate:seq:rel1}
  A \holtype{pterm} $t$ is related to an \holtype{sterm} $u$ if $t = \holconst{sterm\_to\_pterm}\;u$.
\end{definition}

\begin{definition}\label{def:intermediate:seq:rel2}
  An \holtype{sterm} $u$ is related to a \holtype{pterm} $t$ if $\holconst{pterm\_to\_sterm}\;t = u$.
\end{definition}

\begin{corollary}
  \cref{def:intermediate:seq:rel1,def:intermediate:seq:rel2} are strong structural term relations (§\ref{sec:terms:algebra:thy:preds}).
\end{corollary}

\noindent
Corresponding versions of \cref{thm:intermediate:named:subst} can also be proved for these relations.

\subsection{Compilation, correctness, \& completeness}

At the same time the terms are translated, the rules also have to be converted from type $(\holtype{string}\times \holtype{pterm})\;\holtype{set}$ to $(\holtype{string}\times \holtype{sterm})\;\holtype{list}$.
Fortunately, the same \holconst{ordered\_map} function can be used for that, because \holtypejudgement{\holtype{string}}{\holclass{lexorder}}:
%
\begin{lstlisting}[language=Isabelle]
definition compile :: (string $\times$ pterm) set $\Rightarrow$ (string $\times$ sterm) list where
compile //R// = ordered_map {(//name//, pterm_to_sterm //t//) | (//name//, //t//) $\in$ //R//}
\end{lstlisting}
%
\noindent
This looks very similar to the \holconst{Pabs} case in \holconst{pterm\_to\_sterm} (\cref{code:intermediate:seq:convert}).

\remark*{
  Observe that the \holconst{ordered\_map} function is used twice in this section, but with different type instantiations.
  This is the reason why this compiler phase and the proofs are not parametrized on the concrete sequentialization strategy, because locales in Isabelle do not allow polymorphic parameters.
}

\noindent
Luckily, both the correctness and the completeness property of this phase are easily stated and proved:

\begin{theorem}[Correctness]\label{thm:intermediate:seq:correctness}
  \[
    \inferrule*{
      \rewrite{\holconst{compile}\;\R}{t}{u}
    }{\rewrite\R{\holconst{sterm\_to\_pterm}\;t}{\holconst{sterm\_to\_pterm}\;u}}
  \]
\end{theorem}

\begin{proof}
  By rule induction on the new semantics (\cref{code:intermediate:seq}).
  As described in the introduction, any step taken in the sequential semantics can trivially be taken in the unordered semantics too.
\end{proof}

\begin{theorem}[Completeness]\label{thm:intermediate:seq:completeness}
  \[
    \inferrule*{
      \rewrite\R{t}{u}
    }{\rewrite{\holconst{compile}\;\R}{\holconst{pterm\_to\_sterm}\;t}{\holconst{pterm\_to\_sterm}\;u}}
  \]
\end{theorem}

\begin{proof}
  By rule induction on the previous semantics (\cref{code:intermediate:elim:explicit}).
  The challenge here lies within the \textsc{Beta} case, where it needs to be proved that the arbitrary step taken by the previous semantics also corresponds to the first matching clause in the ordered term.
  But this is guaranteed by \cref{thm:terms:algebra:thy:find-match-compat}.
\end{proof}