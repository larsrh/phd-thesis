% !TeX spellcheck = en_GB
% !TeX encoding = UTF-8

\section{Composition}

After having described the complete compiler pipeline in the previous sections, I will give a high-level overview of how all phases fit together here.

The actual compiler can be characterized with two functions.
Firstly, the translation of \holtype{term} to \holtype{Cake.exp} is a simple composition of each term translation function:
%
\begin{lstlisting}[language=Isabelle]
definition term_to_cake :: term $\Rightarrow$ Cake.exp where
term_to_cake = sterm_to_cake $\circ$ pterm_to_sterm $\circ$ nterm_to_pterm $\circ$ term_to_nterm
\end{lstlisting}
%
Secondly, the function that translates function definitions by composing the phases as outlined in \cref{fig:intermediate}, including iterated application of pattern elimination:
\begin{lstlisting}[language=Isabelle]
definition compile :: (term $\times$ term) set $\Rightarrow$ Cake.dec where
compile = Cake.Dletrec $\circ$ compile_srules_to_cake $\circ$ compile_prules_to_srules $\circ$
  compile_irules_to_srules $\circ$ compile_irules_iter $\circ$ compile_crules_to_irules $\circ$
  consts_of $\circ$ compile_rules_to_nrules
\end{lstlisting}
Each function \holconst{compile\_}* corresponds to one compiler phase;
the remaining functions are trivial.

This produces a CakeML top-level declaration.
I prove that evaluating this declaration in the top-level semantics (\holconst{evaluate\_prog}) results in an environment \holconst{cake\_sem\_env}.
But \holconst{cake\_sem\_env} can also be computed via another instance of the global clause set trick~(§\ref{sec:intermediate:value:sublocale}).

Correctness is justified for each phase between intermediate semantics and correspondence relations, most of which are rather technical.
Whereas the compiler may be complex and impenetrable, the trustworthiness of the constructions hinges on the obviousness of those correspondence relations.

Fortunately, under the assumption that terms to be evaluated and the resulting values do not contain abstractions -- or closures, respectively -- all of the correspondence relations collapse to simple structural equality.

Equipped with these functions and relations, I can state the final correctness theorem:
%
\begin{lstlisting}[language=Isabelle]
theorem compiled_correct:
  (* If CakeML evaluation of a term succeeds ... *)
  assumes evaluate False cake_sem_env //s// (term_to_cake //t//) (//s'//, Rval //v//)
  (* ... producing a constructor term without closures ... *)
  assumes cake_no_abs //v//
  (* ... and some syntactic properties of the involved terms hold ... *)
  assumes closed t and $\neg$ shadows_consts //t// and welldefined //t// and wellformed //t//
  (* ... then this evaluation can be reproduced in the term-rewriting semantics *)
  shows //rs// $\vdash$ //t// $\rightarrow^*$ cake_to_term //v//
\end{lstlisting}

\noindent
This theorem directly relates the evaluation of a term $t$ in the full CakeML (including mutability and exceptions) to the evaluation in the initial higher-order term rewriting semantics.
The evaluation of $t$ happens using the environment produced from the initial rule set.
Hence, the theorem can be interpreted as the correctness of the pseudo-ML expression $\holkeyword{let\ rec}\;\rs\;\holkeyword{in}\;t$.

Observe that in the assumption, the conversion goes from my terms to CakeML expressions, whereas in the conclusion, the conversion goes the opposite direction.