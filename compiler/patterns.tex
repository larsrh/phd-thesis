% !TeX spellcheck = en_GB
% !TeX encoding = UTF-8

\section{Explicit pattern matching}
\label{sec:intermediate:elim}

In this step, rule sets are transformed from implicit to explicit pattern matching (§\ref{sec:terms:types:pterm}).
This is an iterative algorithm that requires two fundamental invariants:
all defining equations of one function must
\begin{enumerate}
  \item have the same number of parameters (guaranteed by the \holcommand{function} command during preprocessing) and
  \item be pattern-compatible (\cref{def:preproc:compatibility})
\end{enumerate}
If either invariant is violated, the result of an iteration is unspecified; therefore, they form additional assumptions (§\ref{sec:intermediate:wellformed}).

Defining equations after elimination of implicit patterns will have the form $\embed{\holconst{f}} = \Lambda\;\C$, where \C{} is a set of clauses.
The right-hand side contains $n$ nested abstractions, where $n$ is the number of parameters of the defining equations of \holconst{f}.

The implementation strategy requires successive elimination of a single parameter from right to left, in a similar fashion as Slind's pattern matching compiler \cite[§3.3.1]{slind1999terminating}.
There is an alternative implementation that will be discussed in §\ref{sec:intermediate:elim:discussion}.
Explanation of the semantics in this phase is dependent upon understanding of the elimination procedure, which is why I defer this to §\ref{sec:intermediate:elim:impl}.

The running example in this section is the \holconst{map} function from §\ref{sec:deep:operation}.
It has arity 2.
I omit the brackets \embed{} for brevity.
Recall its definition:
\[
  \begin{array}{lllcl}
    \holconst{map}&f&[] &=& [] \\
    \holconst{map}&f&(x \cons \mathit{xs}) &=& f\;x \cons \holconst{map}\;f\;\mathit{xs}
  \end{array}
\]
%
First, the list parameter is eliminated:
%
\begin{align*}
  \holconst{map}\;f = \lambda\; &[] \Rightarrow [] \\
  |\;&x \cons \mathit{xs} \Rightarrow f\;x \cons \holconst{map}\;f\;\mathit{xs}
\end{align*}
%
Finally, the function parameter is eliminated:
%
\begin{align*}
  \holconst{map} = \lambda\; f \Rightarrow \big(\lambda\; &[] \Rightarrow [] \\
  |\;&x \cons \mathit{xs} \Rightarrow f\;x \cons \holconst{map}\;f\;\mathit{xs}\big)
\end{align*}
%
This final equation has arity zero and is defined by a twice-nested abstraction.

From this example, it becomes obvious why the first invariant has to hold:
the elimination procedure would get stuck for jagged arrays.
The second invariant is more subtle.
Consider an equivalent definition of \holconst{map}, where the function parameter has a different name in the second equation:
%
\[
  \begin{array}{lllcl}
    \holconst{map}&f&[] &=& [] \\
    \holconst{map}&g&(x \cons \mathit{xs}) &=& g\;x \cons \holconst{map}\;g\;\mathit{xs}
  \end{array}
\]
%
Through elimination, this would turn into:
%
\begin{align*}
\holconst{map} = \lambda\; &f \Rightarrow \big(\lambda\; [] \Rightarrow []\big) \\
|\;&g \Rightarrow \big(\lambda\; x \cons xs \Rightarrow f\;x \cons \holconst{map}\;f\;xs\big)
\end{align*}
%
Even though the original equations were non-overlapping, the resulting abstraction has overlapping patterns.
Slind observed a similar problem \cite[§3.3.2]{slind1999terminating} in his algorithm.
Therefore, he only permits \emph{uniform} equations, as defined by Wadler \cite[§5.5]{peytonjones1987implementation}.
In the formalization, I am able to give a formal characterization of the requirements as a computable function on pairs of patterns (§\ref{sec:terms:algebra:thy:matching}).
This compatibility constraint ensures that any two overlapping patterns at the same parameter position are equal and are thus appropriately grouped together in the elimination procedure.

While this rules out some theoretically possible pattern combinations (see also §\ref{sec:preproc:limitations}), in practice, I have not found this to be a problem.
If a function's parameters cannot be renamed accordingly, users can transform a set of equations into a single equation using \holkeyword{case} combinators with the \holcommand{case\_of\_simps} command by Noschinski and Klein.\thyrefdist{HOL-Library}{Simps_Case_Conv}

\subsection{Elimination procedure}
\label{sec:intermediate:elim:proc}

The elimination procedure can be described as an iterative matrix transformation.
Functions are processed one at a time.
In a similar fashion as Slind's pattern matching compiler \cite[§3.3.1]{slind1999terminating}, I view the set of defining equations of \holconst{c} as a matrix where each row represents an equation.
%
\[
  \begin{array}{lccl}
    \holconst{c} & p_{1,1}\;\ldots\;p_{1,n} &=& \mathit{rhs_1} \\
    \holconst{c} & p_{2,1}\;\ldots\;p_{2,n} &=& \mathit{rhs_2} \\
    &&\vdots& \\
    \holconst{c} & p_{m,1}\;\ldots\;p_{m,n} &=& \mathit{rhs_m}
  \end{array}
  \;
  \rightsquigarrow
  \quad
  \left(
    \begin{array}{cccc|c}
    p_{1,1} & p_{1,2} & \cdots & p_{1,n} & \mathit{rhs}_1 \\
    p_{2,1} & p_{2,2} & \cdots & p_{2,n} & \mathit{rhs}_2 \\
    \vdots & \vdots & \ddots & \vdots & \vdots \\
    p_{m,1} & p_{m,2} & \cdots & p_{m,n} & \mathit{rhs}_m
    \end{array}
  \right)
\]
%
\begin{definition}[Arity]
  The \keyword{arity} of a function is the unique number of patterns in each of its defining equations, if it exists, or unspecified otherwise.
  Two equations are \emph{arity-compatible}\index{arity compatibility} if they define different functions or have the same number of patterns.
\end{definition}

\noindent
Applied to the above example, the arity of \holconst{c} is $n$.

In each step, the equations are grouped by the initial $n-1$ patterns (the \emph{prefix}).
More formally, I treat a row in the matrix as an $(n+1)$-tuple $(p_{i,1}, \ldots, p_{i,n}, \mathit{rhs}_i)$ and define an equivalence relation $\equiv_p$ such that
%
\[ (p_{i,1}, \ldots, p_{i,n}, \mathit{rhs}_i) \equiv_p (p_{j,1}, \ldots, p_{j,n}, \mathit{rhs}_j) \biimplies (p_{i,1}, \ldots, p_{i,n-1}) = (p_{j,1}, \ldots, p_{j,n-1}) \]
%
The new matrix is constructed from the set of equivalence classes.
It consists of $n$ columns and one row per equivalence class.
The first $n-1$ columns contain the unique $p_{i,1},\ldots,p_{i,n-1}$, whereas the last column is an abstraction with the set $S_i$ of all $(p_{k,n}, \mathit{rhs}_k)$ in the equivalence class as the set of clauses.
%
\[
  \left(
    \begin{array}{cccc|c}
    p'_{1,1} & p'_{1,2} & \cdots & p'_{1,n-1} & \Lambda\;S_1 \\
    p'_{2,1} & p'_{2,2} & \cdots & p'_{2,n-1} & \Lambda\;S_2 \\
    \vdots & \vdots & \ddots & \vdots & \vdots \\
    p'_{m',1} & p'_{m',2} & \cdots & p'_{m',n-1} & \Lambda\;S_{m'}
    \end{array}
  \right)
\]

\begin{lemma}[Invariant]
  Given a set of arity- and pattern-compatible equations, the elimination procedure produces another set of arity- and pattern-compatible equations.
\end{lemma}

\begin{proof}
  Consider a set of equations with the same head \holconst{c}.
  Assume arity of \holconst{c} is nonzero.
  \begin{enumerate}
    \item
      Obvious from the matrix construction.
    \item
      Let $\holtypejudgement{\R_{\holconst{c}}}{(\holtype{term}\;\holtype{list} \times \holtype{pterm})\;\holtype{set}}$ be all equations for \holconst{c}.
      Also, let $\R_{\holconst{c}}'$ be the set after a single elimination.
      To establish pattern compatibility of a set, pick any two equations $(\mathit{ps}_1, \mathit{rhs}_1)$ and $(\mathit{ps}_2, \mathit{rhs}_2)$ in $\R_{\holconst{c}}'$ and prove their pattern compatibility.
      Consider the origin of these two equations:
      There must be (at least) one equation $(\mathit{ps}_i \append [p_i'], \mathit{rhs}_i') \in \R_{\holconst{c}}$ that gave rise to $(\mathit{ps}_1, \mathit{rhs}_1)$ for $i \in \{1, 2\}$.
      Consequently, using the invariant for $\R_\holconst{c}$, $\mathit{ps}_1 \append [p_1']$ and $\mathit{ps}_2 \append [p_2']$ must be compatible.

      It remains to be proved that $\mathit{ps}_1$ and $\mathit{ps}_2$ are compatible.
      This follows directly from the definition of \holconst{rev\_accum\_rel} (§\ref{sec:terms:algebra:thy:matching}).
      \qedhere
  \end{enumerate}
\end{proof}

\begin{lemma}[Termination]
  For each function in a rule set \R, its arity is zero or decreases by one after a single elimination step.
\end{lemma}

\begin{corollary}\label{thm:intermediate:elim:iter}
  Let \holconst{c} be the function with the maximum arity $n$ in a rule set \R.
  Eliminating patterns in $\R$ for $n$ times yields a rule set $\R'$ where each function has arity zero.
\end{corollary}

\subsection{Compilation}
\label{sec:intermediate:elim:impl}

\begin{code}[t]
  \begin{subcode}{\linewidth}
    \[
      \inferrule*[left=Step]{(\mathit{name}, \C) \in \R \\ ([p_1, \ldots, p_n], \mathit{rhs}) \in C \\ \holconst{match}\;(\holconst{Pconst}\;\mathit{name}\mathbin\$p_1\mathbin\$\ldots\mathbin\$p_n)\;t = \holconst{Some}\;\sigma}{\rewrite{\R}{t}{\holconst{subst}\;\sigma\;\mathit{rhs}}}
    \]
    \[
      \inferrule*[left=Beta]{(\mathit{pat}, \mathit{rhs}) \in \C \\ \holconst{match}\;\mathit{pat}\;t = \holconst{Some}\;\sigma \\ \holconst{closed}\;t}{\rewrite{\R}{\app{(\Lambda\; \C)}{t}}{\holconst{subst}\;\sigma\;\mathit{rhs}}}
    \]
    \[
      \inferrule*[left=Fun]{\rewrite{\R}{t}{t'}}{\rewrite{\R}{\app t u}{\app{t'}{u}}}
      \quad
      \inferrule*[left=Arg]{\rewrite{\R}{u}{u'}}{\rewrite{\R}{\app t u}{\app{t}{u'}}}
    \]
    \caption{Combined implicit and explicit pattern matching}
  \end{subcode}
  \vspace{1em}

  \begin{subcode}{\linewidth}
    \[
      \inferrule*[left=Step']{(\mathit{name}, \mathit{rhs}) \in \R}{\rewrite{\R}{\holconst{Pconst}\;\mathit{name}}{\mathit{rhs}}}
    \]
    \caption{Modified \textsc{Step} rule for explicit-only pattern matching}
    \label{code:intermediate:elim:explicit}
  \end{subcode}

  \caption{Small-step semantics with pattern matching}
  \label{code:intermediate:elim}
\end{code}

After transformation to \holtype{nterm}, equations are represented as a tuple $\holtype{term} \times \holtype{nterm}$.
But recall from \cref{fig:intermediate} that the type of $\R$ in this phase is $(\holtype{string} \times (\holtype{term}\;\holtype{list} \times \holtype{pterm})\;\holtype{set})\;\holtype{set}$.
The full compilation goes through multiple sub-phases:

\begin{enumerate}
  \item
    The left-hand side of each equation (of type \holtype{term}) is destructured into a tuple $\holtypejudgement{(\mathit{name}, \allowbreak\mathit{pats})\allowbreak}{\holtype{string} \times \holtype{term}\;\holtype{list}}$, where $\mathit{name}$ represents the head of the equation and $\mathit{pats}$ the list of patterns.
  \item
    The right-hand side (of type \holtype{nterm}) can be trivially embedded into \holtype{pterm}:
    An \holtype{nterm}-abstraction $\Lambda x.\;t$ is translated to the \holtype{pterm}-abstraction $\Lambda \{\embed x \Rightarrow t\}$, i.e., a case abstraction with the single clause $(\holconst{Pvar}\;x, t)$.
  \item
    Equations with the same head are grouped together.
    Different groups are processed separately.
    A single group has the type $(\holtype{term}\;\holtype{list} \times \holtype{pterm})\;\holtype{set}$.
    This corresponds to the initial matrix representation as depicted in §\ref{sec:intermediate:elim:proc}.
  \item
    For each function (i.e.\ for each group) that has an arity greater than zero, the elimination procedure is applied.
    The type does not change here.
    This can be iterated until all functions have arity zero (\cref{thm:intermediate:elim:iter}).
\end{enumerate}
%
For correctness and completeness purposes, all but the last step have purely syntactic proofs, that is, they do not require arguing about the semantics.

The target semantics has two variants and is given in \cref{code:intermediate:elim}.
The first variant with the \textsc{Step} rule performs both implicit and explicit pattern matching, to account for functions that may have nonzero arity.

The second variant applies after a post-processing step.
When all functions in the rule set are of arity zero, the implicit pattern lists are all empty.
Consequently, the rule set can be trivially converted to type $(\holtype{string} \times \holtype{pterm})\;\holtype{set}$ by stripping the pattern lists.
To accommodate for this, the modified \textsc{Step'} rule merely replaces a constant by its definition, without taking arguments into account.
Only the \textsc{Beta} rule performs pattern matching.

The remainder of this section is concerned with the proofs for the elimination procedure.
I will briefly revisit the initial and post-processing steps in §\ref{sec:intermediate:elim:discussion}.

\subsection{Correspondence relation}
\label{sec:intermediate:elim:corr}


\begin{code}[t]
  \[
    \inferrule*[left=Const]{ }{\corrp{\holconst{Pconst}\;n}{\holconst{Pconst}\;n}} \quad
    \inferrule*[left=Var]{ }{\corrp{\holconst{Pvar}\;n}{\holconst{Pvar}\;n}} \quad
    \inferrule*[left=Comb]{\corrp{t_1}{t_2} \\ \corrp{u_1}{u_2}}{\corrp{\app{t_1}{u_1}}{\app{t_2}{u_2}}}
  \]
  \[
    \inferrule*[left=Ext]{
      \holconst{rel}\;(\lambda (p_1, t_1)\;(p_2, t_2).\; p_1 = p_2 \wedge \corrp{t_1}{t_2})\;\C_1\;\C_2
    }{\corrp{\Lambda\;\C_1}{\Lambda\;\C_2}}
  \]
  \[
    \inferrule*[left=Defer]{
      \forall i.\;\holconst{closed}\;t_i \\
      (f, \R_f) \in \R \\
      \holconst{arity}\;\R_f > 0 \\
      \holconst{rel}\;(\lambda (p_1, t_1)\;(p_2, t_2).\; p_1 = p_2 \wedge \corrp{t_1}{t_2})\;(\holconst{deferred}\;[t_1, \ldots, t_n]\;\R_f)\;\C
    }{\corrp{\holconst{Pconst}\;f\mathbin\$t_1\mathbin\$\ldots\mathbin\$t_n}{\Lambda\;\C}}
  \]
  \begin{align*}
    \holconst{deferred}\;\mathit{ts}\;R_f = \{(p_{n+1}, &\holconst{subst}\;\sigma\;\mathit{rhs}) \mid\\
    &([p_1, p_2, \ldots, p_{n+1}], \mathit{rhs}) \in \R_f \wedge \holconst{matchs}\;[p_1, \ldots, p_n]\;\mathit{ts} = \holconst{Some}\;\sigma \}
  \end{align*}
  \caption{Left-deferred correspondence}
  \label{code:intermediate:elim:corr}
\end{code}

The statement of the semantic correctness property is more difficult than in the previous phase.
The obvious property does not hold:
%
\[
  \inferrule{
    \rewrite{\holconst{compile}\;\R}{t}{u} \\
    \holconst{closed}\;t
  }{
    \rewrite\R t u
  }
\]
%
Consider the \holconst{map} function again.
After eliminating once, the defining equation of \holconst{map} is of the form $\embed{\holconst{map}\;f} = \Lambda\;C$, which means that the term $\embed{\holconst{map}\;\holconst{id}}$ can be rewritten to $\Lambda\;C$.
However, this rewrite step cannot be taken in the original rule set, because the second argument is missing.

Because of the absence of a direct correspondence, I have introduced a relation $\approx_\text{p}$ (\cref{sec:intermediate:elim:corr}).
The ultimate goal is that the following correctness property holds:
%
\[
  \inferrule{
    \rewrite{\holconst{compile\_single}\;\R}{u}{u'} \\
    \corrp{t}{u} \\
    \holconst{closed}\;t
  }{
    \exists t'.\;\rewrite*\R t t' \wedge \corrp{t'}{u'}
  }
\]
%
where \holconst{compile\_single} is a single application of the elimination procedure.
For this to work, $\approx_\text{p}$ must take the rule set into account.

I will illustrate the meaning of this relation based on the \holconst{map} example.
Consider its matrix representation:
\[
  R_{\holconst{map}} =
    \left(\begin{array}{cc|c}
      \embed f & \embed{[]} & \embed{[]} \\
      \embed f & \embed{x \cons \mathit{xs}} & \embed{f\;x \cons \holconst{map}\;f\;\mathit{xs}}
    \end{array}\right)
\]
%
After an elimination step:
%
\[
  R'_{\holconst{map}} =
    \left(\begin{array}{c|c}
      \embed f &
        \embed{\begin{aligned}
                 \lambda\; &[] \Rightarrow [] \\
                 |\;&x \cons \mathit{xs} \Rightarrow (\lambda y.\;y)\;x \cons \holconst{map}\;(\lambda y.\;y)\;\mathit{xs}
               \end{aligned}}
    \end{array}\right)
\]
%
In this modified rule set, the term $\embed{\holconst{map}\;(\lambda y.\;y)}$ can be rewritten.
In the original rule set, it cannot.
I refer to such a rewriting as a \emph{half-step}.
Using the \textsc{Defer} rule, the un-reduced and the reduced term are related:
%
\[
  \corrp{
    \embed{\holconst{map}\;(\lambda y.\;y)}
  }{
    \embed{\begin{aligned}
      \lambda\; &[] \Rightarrow [] \\
      |\;&x \cons \mathit{xs} \Rightarrow (\lambda y.\;y)\;x \cons \holconst{map}\;(\lambda y.\;y)\;\mathit{xs}
    \end{aligned}}
  }
\]
%
The \textsc{Defer} rule can be explained by examining the \holconst{deferred} function.
Given a function application for $n$ parameters $\embed{\holconst{c}\;t_1\;\ldots\;t_n}$ of a function with arity $n+1$, it selects all defining equations that so far match these arguments (minus the $n+1$-st one).
Each of these equations $\embed{\holconst{c}\;p_1\;\ldots\;p_n\;p_{n+1} = t}$ carries an additional pattern $p_{n+1}$ for the $(n+1)$st argument which has to be supplied eventually.
From these equations, I construct a $\Lambda$-abstraction comprising pairs $(\embed{p_{n+1}}, \holconst{subst}\;\sigma\embed t)$, where $\sigma$ is the result of matching the initial $n$ patterns.

Based on this understanding, the $\approx_\text{p}$ can be described as a \emph{left-deferred} or a \emph{right-extensional} correspondence.

In the case of abstraction-free terms, the relation collapses to equality:

\begin{lemma}\label{thm:intermediate:elim:eq}
  If $\corrp t u$ and $u$ is abstraction-free, then $t = u$.
\end{lemma}

\begin{proof}
  By rule induction on $\approx_\text{p}$.
  Neither the \textsc{Ext} nor the \textsc{Defer} rule are applicable; consequently, only \textsc{Const}, \textsc{Var}, and \textsc{Comb} are left.
  They describe an equality relation.
\end{proof}

\noindent
This lemma becomes important when considering the iterated elimination, because $\approx_\text{p}$ is not transitive.

\begin{lemma}\label{thm:intermediate:elim:rel-struct}
  $\approx_\text{p}$ is reflexive. $\approx_\text{p}$ is a structural term relation (§\ref{sec:terms:algebra:thy:preds}).
\end{lemma}

\remark{
  $\approx_\text{p}$ is an example of a structural term relation that is not also a strong structural relation:
  If $t$ and $u$ are related through the \textsc{Defer} rule, then $t$ may be an application or a constant, whereas $u$ is an abstraction, hence having different shapes.
  Consequently, \cref{thm:terms:algebra:match-rel} would be violated.
}

\begin{lemma}[Substitution]\label{thm:intermediate:elim:subst}
  Let $t_1$ and $t_2$ be related terms.
  Also, let $\sigma_1$ and $\sigma_2$ be closed and related environments.
  Then, the results of substitution are also related.
  Formally: $\corrp{\holconst{subst}\;\sigma_1\;t_1}{\holconst{subst}\;\sigma_2\;t_2}$.
\end{lemma}

\subsection{Correctness}

Recall the desired correctness property as stated in §\ref{sec:intermediate:elim:corr}:

\begin{theorem}\label{thm:intermediate:elim:correctness}
  \[
    \inferrule{
      \rewrite{\holconst{compile\_single}\;\R}{u}{u'} \\
      \corrp{t}{u} \\
      \holconst{closed}\;t
    }{
      \exists t'.\;\rewrite*\R t t' \wedge \corrp{t'}{u'}
    }
  \]
\end{theorem}

\noindent
Observe that the conclusion uses the starred rewrite relation, i.e., its reflexive-transitive closure.
The reason is simple:
one step in the transformed rule set may correspond to zero or one step in the original rule set.

\begin{proof}
  By rule induction over the semantics (\cref{code:intermediate:elim}).
  Similarly to the proof of \cref{thm:intermediate:named}, the interesting cases are \textsc{Step} and \textsc{Beta}.

  \begin{induction}
    \item[Step]
      In this case, $u$ is an application of a constant to a number of arguments.
      It holds that $\rewrite{\holconst{compile\_single}\;R}{\embed{\holconst{c}\;u_1\;\ldots\;u_n}}u'$ for some $u'$ and $u_i$.
      Case distinction on \holconst{c}'s arity in \R:

      \begin{itemize}
        \item
          If \holconst{c} already had arity zero, nothing changed during compilation.
          In particular, there is a rule in $\R$ that matches and can rewrite the function application to $u'$.
          The same rewrite step needs to be taken from a $t$ where $\corrp{t}{u}$.
          This follows from \cref{thm:intermediate:elim:subst,thm:intermediate:elim:rel-struct}.
          Consequently, $\rewrite\R{\embed{\holconst{c}\;t_1\;\ldots\;t_n}}t'$ where $\corrp{t'}{u'}$ with a single rewrite step.
        \item
          If not, the arity of \holconst{c} decreased.
          This case requires establishing the correctness of the \holconst{deferred} set:
          the transformed rule set is able to perform a half-step, whereas $\R$ cannot.
          Hence, set $t' = t$ with $\rewrite*\R{t}{t'}$ because of the reflexive-transitive closure.
          It remains to be shown that $\corrp{t}{u'}$.
          This requires \cref{thm:intermediate:elim:subst,thm:terms:algebra:compatibles-eq}.
      \end{itemize}
    \item[Beta]
      In this case, $u$ is an application where the function is a case abstraction, i.e., there is an $x$ such that $u = \app{(\Lambda\;C)}{x}$ and $\rewrite{\holconst{compile\_single}\;R}{(\Lambda\;C)\;x}{u'}$ for some $u'$.
      $u'$ is the result of matching $x$ to a pattern $p$ in $C$ and subsequent substitution.
      Now, $t$ is a term such that $\corrp{t}{\app{(\Lambda\;C)}{x}}$.
      Using the \textsc{Comb} rule, there must be $t_1$ and $t_2$ with $t = \app{t_1}{t_2}$ and $\corrp{t_1}{\Lambda\;C}$ and $\corrp{t_2}{x}$.
      There are two possible cases in $\approx_\text{p}$ that could give rise to $\corrp{t_1}{\Lambda\;C}$:
      \begin{semantics}
        \item[Ext]
          This is the simple case, because it reveals that $t_1$ has an identical structure to $\Lambda\;C$.
          Hence, there is an equivalent pattern $p'$ in $t_1$ as in $C$.
          The result follows from \cref{thm:intermediate:elim:subst,thm:intermediate:elim:rel-struct}.
        \item[Defer]
          The situation is that $t_1$ is an application of a constant to a list of arguments.
          The rewrite step could not be completed because the number of arguments is one less than the arity of that constant.
          However, since $t = \app{t_1}{t_2}$, the step can now be completed, because $t_2$ is available as an extra argument.
          On the other side, $\Lambda\;C$ are precisely the clauses that have been deferred.
          It now remains to find the matching pattern $p$ in $C$ and establish the existence of a corresponding equation in \R.
          The technical challenge is that the right-hand sides in $C$ have already been partially substituted by the half-step.
          This can be solved by using \cref{thm:terms:algebra:match-matchs,thm:terms:algebra:subst-indep}, which allow to split the matching and substitution into two parts: first the $n$-ary prefix and then the additional pattern $p$. \qedhere
      \end{semantics}
  \end{induction}
\end{proof}

\noindent
Combining this result with \cref{thm:intermediate:elim:eq} yields:

\begin{corollary}[Correctness for abstraction-free results]
  \[
    \inferrule{
      \rewrite{\holconst{compile\_single}\;\R}{t}{u} \\
      \holconst{closed}\;t \\
      \holconst{no\_abs}\;u
    }{
      \rewrite*\R t u
    }
  \]
\end{corollary}

\noindent
This corollary can be lifted to \holconst{compile}, i.e., the iterated application of \holconst{compile\_single}.

\subsection{Completeness}

\authorship{
  The proof in this section has been contributed by Yu Zhang.
}

\noindent
The opposite direction can be stated directly, without the use of an additional relation:

\begin{theorem}\label{thm:intermediate:elim:completeness}
  \[
    \inferrule{
      \rewrite\R t t' \\
      \holconst{closed}\;t
    }{
      \rewrite*{\holconst{compile\_single}\;\R}{t}{t'}
    }
  \]
\end{theorem}

\noindent
Whereas the correctness requires the original semantics to perform zero or one step, completeness requires the new semantics to perform one or two steps:
if the arity of one constant has been reduced during pattern elimination, both half-steps need to be executed in order.
This is also the idea of the proof for the \textsc{Step} case; the other cases are trivial.

\subsection{Discussion}
\label{sec:intermediate:elim:discussion}

This compilation phase is both non-trivial and has some minor restrictions on the set of function definitions that can be processed.
Instead of eliminating patterns from right to left, patterns could also alternatively be grouped into tuples.
The \holconst{map} example would be translated into:
%
\begin{align*}
  \holconst{map} = \lambda\; &(f, []) \Rightarrow [] \\
  |\;&(f, x \cons \mathit{xs}) \Rightarrow f\;x \cons \holconst{map}\;f\;\mathit{xs}
\end{align*}
%
The compilation of patterns would then be left for the CakeML compiler, which has no pattern compatibility restriction.

Despite the simpler idea behind the algorithm, there are two disadvantages:
\begin{itemize}
  \item the compiler phase would require the knowledge of a tuple type in the term language, which is otherwise unaware of concrete datatypes
  \item the correspondence relation would be harder to specify, because the alternative translation goes directly from arity $n$ to $0$.
\end{itemize}

\noindent
Finally, I will review the complexity of the proofs in this section.
Recall the detailed structure of this compiler phase as outlined in §\ref{sec:intermediate:elim:impl}.

The vast majority of the complexity is caused by the iterative elimination procedure.
The other parts, namely the massaging of \holtype{nterm}s to arrive at a syntactically equivalent rule set of \holtype{pterm}s, and similarly to remove empty implicit pattern lists, only require some basic reasoning on sets.
But all steps share the requirement for classical logic.
Frequently, I have resorted to using choice operators, for example, when determining the arity of a function definition.
Nonetheless, the entire routine is executable (§\ref{sec:background:code}).